\documentclass[a4paper,11pt]{article}
\usepackage{ajmesa}
\usepackage[top = 1in, left = 0.5in, right = 0.5in, bottom = 1in]{geometry}

\pagestyle{fancy}
\fancyhf{}
%\rhead{Math 40 - WFY | Julius Basilla}
%\chead{2019-04243}
\lhead{EEE 133 Key concepts and equations}
\cfoot{Page \thepage \hspace{1pt} of \pageref{LastPage}}

\title{\textbf{EEE 133 Key concepts and Equations}}
\author{AJ Mesa Jr.}

\begin{document}
	\maketitle
	\section*{Differential Equations}
	A linear ordinary differential equation of constant coefficients follows the form 
	\begin{equation}
		a_n\diff[n]{x}{t} + a_{n - 1}\diff[n - 1]{x}{t} + \ldots + a_1\diff{x}{t} + a_0x = f(t)
	\end{equation}
	A homogenous differential equation is an equation where the independent variable appears to have the same power. Equation (1) is homogenous when $f(t) = 0$.\\ 
	The solution to the linear ODE in (1) has the form
	\begin{equation}
		x(t) = x_h(t) + x_p(t)
	\end{equation}
	Where $x_h(t)$ is a solution to the homogenous equation while $x_p(t)$ is a solution to the nonhomogenous equation.  \\
	A homogenous equation has solutions of the form
	\begin{equation}
		x_h(t) = \exp(mt)
	\end{equation}
	for $m$ is a root of the equation
	\begin{equation}
		m^n + \frac{a_{n - 1}}{a_{n}}m^{n - 1} + \ldots + \frac{a_{1}}{a_{n}}m + \frac{a_{0}}{a_{n}} = 0
	\end{equation}
	If a root $m_i$ is repeated $k$ times, the corresponding solutions are $\exp(m_i t),~t\exp(m_it),~\ldots,~t^{k - 1}\exp(m_it)$
	
	\subsection*{Some common forms (with constant coefficients) and solutions}
	\begin{center}
	\begin{tabular}{| c | c | c | c |}
		\hline
		Equation & Characteristic Equation & Determinant & Solution \\ \hline
		$a_1\diff{x}{t} + a_0x = 0$ & $a_1m + a_0 = 0$ & & $x = C\exp\left(-\frac{a_0}{a_1}t\right)$ \\ \hline 
		& \multirow{3}{*}{$a_2m^2 + a_1m + a_0 = 0$} & $a_1^2 - 4a_2a_0 > 0$ & $x(t) = C_1\exp(m_1t) + C_2\exp(m_2t)$ \\ \cline{3-4}
		$a_2\diff[2]{x}{t} + a_1\diff{x}{t} + a_0x = 0$ & & $a_1^2 - 4a_2a_0 = 0$ & $x(t) = (C_1x + C_2)\exp(m_1t)$ \\ \cline{3-4}
		& & $a_1^2 - 4a_2a_0 < 0$ & $x(t) = \exp(\alpha t)\left[C_1\cos(\beta t) + C_2\sin(\beta t)\right]$ \\ \hline
	\end{tabular}
	\end{center}
	In the last line, the characteristic equation has solutions $m_1 = \alpha + j\beta,~m_2 = \alpha - j\beta$ where $j^2 = -1$
	
	\section*{Laplace Transforms and Theorems}
	A Laplace transform of a function $f(t)$ defined for all $t \geq 0$ is the transformation $$\mathcal{L}[f(t)] = F(s) = \int\limits_0^\infty f(t)\exp(-st) \dd t$$
	An inverse Laplace transform of a function $F(s)$ is defined as:
	$$f(t) = \mathcal{L}^{-1}[F(s)] = \frac{1}{2\pi j} \lim_{T \to \infty} \int\limits_{\gamma - jT}^{\gamma + jT} \exp(st)F(s)\dd s$$
	
	\subsection*{Common Functional Tranforms}
	\begin{center}
	\begin{tabular}{| L | L | L | L |}
		\hline
		f(t) = \mathcal{L}^{-1}[F(s)] & F(s) = \mathcal{L}[F(s)] & f(t) = \mathcal{L}^{-1}[F(s)] & F(s) = \mathcal{L}[F(s)] \\ \hline
		\delta(t - a) & \exp \left( - as \right) & \frac{1}{\beta - \alpha}\left(\exp(-\alpha t) - \exp(-\beta t)\right)u(t) & \frac{1}{(s + \alpha)(s + \beta)} \\ \hline
		u(t - a) & \frac{\exp \left(-as \right)}{s} & \sin(\omega t) u(t) & \frac{\omega}{s^2 + \omega^2} \\ \hline
		tu(t) & \frac{1}{s^2} & \cos(\omega t) u(t) & \frac{s}{s^2 + \omega^2} \\ \hline 
		\frac{t^{n - 1}}{(n - 1)!} u(t) & \frac{1}{s^n} & \sin(\omega t + \theta) u(t) & \frac{s\sin(\theta) + \omega\cos(\theta)}{s^2 + \omega^2} \\ \hline
		\exp(-\alpha t) u(t) & \frac{1}{s + \alpha} & \cos(\omega t + \theta) u(t) & \frac{s\cos(\theta) - \omega\sin(\theta)}{s^2 + \omega^2} \\ \hline
		t\exp(-\alpha t)u(t) & \frac{1}{(s + \alpha)^2} & \exp(-\alpha t) \sin(\omega t)u(t) & \frac{\omega}{(s + \alpha)^2 + \omega^2} \\ \hline 
		\frac{t^{n - 1}}{(n - 1)!}\exp(-\alpha t)u(t) & \frac{1}{(s + \alpha)^n} & \exp(-\alpha t) \cos(\omega t)u(t) & \frac{s + \alpha}{(s + \alpha)^2 + \omega^2} \\ \hline 
	\end{tabular}
	\end{center}

	\textbf{Common Operational Transforms}
	\begin{center}
		\begin{longtable}{|c|c|c|}
			\hline
			Operation & $f(t)$ & $F(s)$ \\ \hline
			Multiplication by constant & $cf(t)$ &  $cF(s)$ \\ \hline
			Addition & $f_1(t) + f_2(t) - f_3(t) + \ldots$ & $F_1(s) + F_2(s) - F_3(s) + \ldots$ \\ \hline
			First time derivative & $\diff{f(t)}{t}$ & $sF(s) - f(0^-)$ \\ \hline 
			Second time derivative & $\diff[2]{f(t)}{t}$ & $s^2F(s) - sf(0^-) - \diff{f(0^-)}{t}$ \\ \hline
			$n$th time derivative & $\diff[n]{f(t)}{t}$ & $s^nF(s) - \sum_{i = 1}^{n}s^{n - i}\diff[i - 1]{f(0^-)}{t}$ \\ \hline
			Time integral & $\int\limits_0^t f(x)\dd x$ & $\frac{F(s)}{s}$ \\ \hline
			Translation in time & $f(t - a)u(t - a),~a > 0$ & $\exp(-as)F(s)$ \\ \hline
			Translation in frequency & $\exp(-at)f(t)$ & $F(s + a)$ \\ \hline
			Scale change & $f(at),~a > 0$ & $\frac{1}{a}F\left(\frac{s}{a}\right)$ \\ \hline
			First frequency derivative & $tf(t)$ & $\diff{F(s)}{s}$ \\ \hline
			$n$th frequency derivative & $t^nf(t)$ & $(-1)^n\diff[n]{F(s)}{s}$ \\ \hline
			Frequency integral & $\frac{f(t)}{t}$ & $\int\limits_{0}^{\infty} F(u)\dd u$ \\ \hline
		\end{longtable}
	\end{center}
	
	\section{Passive components}
	Current and Voltage for Passive Components:
	\begin{center}
	\begin{tabular}{| c | c | c | c |}
		\hline
		Component & Current & Voltage & Energy \\ \hline
		Resistor $R$ & $i = \frac{v}{R}$ & $v = iR$ & $w_R = \int iv \dd t$\\ \hline
		Capacitor $C$ & $i(t) = C\diff{v}{t}$ & $v(t) = v(t_0) + \frac{1}{C}\int_{t_0}^t i(t) \dd t$ & $w_C(t) = \frac{1}{2}Cv^2(t) $ \\ \hline
		Inductor $L$ & $i(t) = i(t_0) + \frac{1}{L} \int_{t_0}^t v(t) \dd t$ & $v(t) = L\diff{i}{t}$ & $w_L(t) = \frac{1}{2}Li^2(t)$\\ \hline
	\end{tabular}
	\end{center}
	Combination of passive components: 
	\begin{center}
	\begin{tabular}{| c | c | c |}
		\hline
		Component & Series & Parallel  \\ \hline
		Resistor $R$ & $R_{eq} = \sum_{i} R_i$ & $\frac{1}{R_{eq}} = \sum_{i} \frac{1}{R_i}$ \\ \hline
		Capacitor $C$ & $\frac{1}{C_{eq}} = \sum_{i} \frac{1}{C_i}$ & $C_{eq} = \sum_{i} C_i$  \\ \hline
		Inductor $L$ & $L_{eq} = \sum_{i} L_i$ & $\frac{1}{L_{eq}} = \sum_{i} \frac{1}{L_i}$\\ \hline
	\end{tabular}
	\end{center}
	
	\begin{figure}[!htb]
		\centering
		\begin{minipage}{.5\textwidth}
			\centering 
			\begin{circuitikz}[american]
				%\ctikzset{bipoles/length=1cm}
				\draw (0, 0) node[op amp] (opamp) {}
				(opamp.-) to[short,-*] ++(-1, 0) coordinate(A)
				(opamp.+) -| ++(-1,-1) node[ground](B){}
				(opamp.out) to[short,*-o] ++(1, 0) to[open, v= $v_o$] ++(0,-1.5) node[ground]{}
				(A) to[C,l_=$C_1$,-o] ++(-2, 0) -- ++(-1, 0) coordinate(D) to [V<=$v_s$] (D |- B) node[ground]{}
				(A) |- ++(1,1) coordinate[yshift=1ex] (L1) to[R=$R_2$] ++(2,0) -| (opamp.out) -- ++(1,0)
				;
			\end{circuitikz}
			\caption{Op Amp Differentiator}
			$$v_o(t) = -R_2C_1\diff{v_s}{t}$$
		\end{minipage}%
		\begin{minipage}{0.5\textwidth}
			\centering
			\begin{circuitikz}[american]
				%\ctikzset{bipoles/length=1cm}
				\draw (0, 0) node[op amp] (opamp) {}
				(opamp.-) to[short,-*] ++(-1, 0) coordinate(A)
				(opamp.+) -| ++(-1,-1) node[ground](B){}
				(opamp.out) to[short,*-o] ++(1, 0) coordinate(C)
				(C) to[open, v= $v_o$] ++(0,-1.5) node[ground]{}
				(A) to[R,l_=$R_1$,-o] ++(-2, 0) -- ++(-1, 0) coordinate(D) to [V<=$v_s$] (D |- B) node[ground]{}
				(A) |- ++(1,1) coordinate[yshift=1ex] (L1) to[C=$C_2$] ++(2,0) -| (opamp.out) to[short,-o] ++(1,0)
				;
			\end{circuitikz}
			\caption{Op Amp Integrator}
			$$v_o(t) = -\frac{1}{R_1C_2}\int_0^t v_s(t') \dd t'$$
		\end{minipage}
	\end{figure}	

	\subsection{Equilibrium Equations}
	\begin{enumerate}
		\item \textbf{Loop Current formulation}:
			\begin{itemize}
				\item number of unknown currents equal number of loops
				\item KVL equation for each loop
			\end{itemize}
		\item \textbf{Node Voltage formulation}:
		\begin{itemize}
			\item number of unknown voltage equal number of nodes except reference
			\item KCL equation for each node
		\end{itemize}
	\end{enumerate}
	
	\subsection{First Order Circuits}
	\begin{enumerate}
		\item first order circuits are any circuit with a single energy storage element, an arbitrary number of sources and resistors
		\item any current or voltage in such circuit is a solution to a first order differential equation
		
	\end{enumerate} 

	\section{First order Unforced Response}
	\begin{center}
	\begin{tabular}{|c|c|c|c|c|}
		\hline
		Network & Current & Voltage & Time Constant & DC Steady State \\ \hline
		Sourcefree RL & $i_L(t) = I_0\exp\left(-\frac{R}{L}t\right)$ & $v_L = -RI_0\exp\left(-\frac{R}{L}t\right)$ &$\tau = \frac{L}{R}$ & $v_L = 0$ \\ \hline 
		Sourcefree RC & $i_C(t) = -\frac{V_0}{R}\exp\left(-\frac{1}{RC}t\right)$ & $v_c = V_0\exp\left(-\frac{1}{RC}t\right)$ & $\tau = RC$ & $i_C = 0$ \\ \hline
	\end{tabular}
	\end{center}
	Typical time constant for RL is in ms, for RC in $\mu$s. For general RL and RC circuits, find the equivalent resistance as seen by the inductor/capacitor. 
	
	
	\section{First order Forced Response}
	Consider a series RL circuit with voltage $v(t) = V_0u(t)$. We have the following KVL equation of current
	\begin{align*}
		i(t) &= 0 & t < 0 \\
		Ri + L\diff{i}{t} &= V_0 & t > 0 
	\end{align*}
	This differential equation gives a solution of 
	\begin{equation}\label{key}
		i(t) = \left[\underbrace{\frac{V_0}{R}}_\text{forced response} - \underbrace{\left(\frac{V_0}{R} - I_0\right)\exp\left(-\frac{R}{L}t\right)}_\text{natural response}\right]u(t)
	\end{equation}
	\begin{enumerate}
		\item Forced response
			\begin{itemize}
				\item dependent on forcing function
				\item steady state response $t \gg \tau$
			\end{itemize}
		\item Natural response
			\begin{itemize}
				\item similar to source-free circuit
				\item dependent on initial values and forcing function
				\item transient response	
			\end{itemize}
	\end{enumerate}	
	Generalization:
	\begin{eqnarray}
		i_L(t) &= \left[I_f - (I_f - I_i)\exp\left(-\frac{t}{\tau}\right)\right]u(t) \\
		v_c(t) &= \left[V_f - (V_f - V_i)\exp\left(-\frac{t}{\tau}\right)\right]u(t)
	\end{eqnarray}
	Note that equation (7) is for the voltage across the capacitor in an RC circuit. 
	
	\subsection{Square Waves and Sequentially Switched Circuits}
	Now consider the series RL circuit with voltage $v(t) = V_0u(t) - V_0u(t - t_0)$ and $I_0 = 0$. We can apply superposition to find the current:
	\begin{align*}
		i(t) &= i_1(t) + i_2(t) \\ 
		i(t) &= \underbrace{\left[\frac{V_0}{R} \left( 1 - \exp \left( -\frac{R}{L}t \right) \right) \right]u(t)}_\text{caused by $V_0u(t)$ alone} - \underbrace{\left[\frac{V_0}{R} \left(1 - \exp \left( -\frac{R}{L}(t - t_0) \right) \right) \right]u(t - t_0)}_\text{caused by $V_0u(t - t_0)$ alone}
	\end{align*}
	For sequentially switched circuits, we consider the pulse width (PW) and period (T) of the pulsing. 
	\begin{center}
	\begin{tabular}{|l|l|l|l|}
		\hline
		Condition & Output & Condition & Output \\ \hline
		PW $\gg \tau$ & time enough to fully charge & T - PW $\gg \tau$ & time enough to fully discharge \\ \hline
		PW $\ll \tau$ & time NOT enough to fully charge & T - PW $\ll \tau$ & time NOT enough to fully discharge \\ \hline
	\end{tabular}
	\end{center}
	
	\subsection{RC Oscillator}
	\begin{outline}[enumerate]
		\1 For a low pass $RC$ circuit, the output (voltage at capacitor) is simply the input for low frequencies
			\2 at $\omega = 0$, $v_o = v_i$ 
			\2 at $\omega \to \infty$, $v_o \to 0$
		\begin{figure}[h]
			\centering
			\begin{circuitikz}[american]
				\draw (0,0) node[circ] {} to[R, l_ = $R$] (2,0) node[circ]{}
				(2,0) to[C, l_ = $C$] (2, -2) node[circ]{} node[ground]{}
				(2,0) -- (4,0) node[ocirc] {}
				(2,-2) -- (4,-2) node[ocirc] {}
				(0,-2) -- (4,-2)
				(4,0) node[circ]{} to[open, v = $v_o(t)$] (4,-2) node[circ]{}
				(0,0) to[open, v = $v_i(t)$] (0, -2) node[circ]{}
				;		
			\end{circuitikz}
			\caption{Low Pass Circuit}
		\end{figure}
	
	
		\1 Schmitt Trigger $RC$ 
		\begin{figure}[h]
			\centering
			\begin{circuitikz}[american]
				\draw (0, 0) node[nand port] (nand) {} 
				(nand.in 1) node[circ, label = {above:1}] {} -| ++(-1, 1) node[vcc] (vcc) {$5V$}
				(nand.in 2) node[circ, label = {above:2}] {} -- ++(-1,0) coordinate(A) -- ++(-1,0) coordinate(B)  node[circ] {}
				(B) to [C, l_ = \mbox{$C_1 = 100 pF$}] ++(0,-2) node[ground] {}
				(A) node[circ] {} |- ++(0.5,-1) to[R, l_ = \mbox{$R_{2} = 5.1k\Omega$}] ++(2,0) to[R, l_ = \mbox{$R_{1} = 10k\Omega$}] ++(2,0) coordinate(C)
				(nand.out) -| (C) node[circ] {} -- ++(0,-1)
				(B) to[short] ++(0,2) node[circ, label = {above:Ant}] {}
				;
			\end{circuitikz}
		\end{figure}
			\2 $V_p = 2.9 V \approx 3.0V, V_n = 1.9 V \approx 2.0V$
			\2 As the voltage on $C_1$ reaches $V_p$, the output will become voltage low. Then the voltage across the capacitor decays
			\2 When the voltage across $C_1$ decays to $V_n$, output will become voltage high. Then the voltage across the capacitor increases 
			\2 The behavior oscillates. 
			\2 the voltage across the charging capacitor is 
			\begin{align*}
				v &= V_{cc}\left[ 1 - \exp\left( -\frac{t}{\tau} \right) \right]
			\end{align*}
			\2 the time to charge to $V_p$ is 
			\begin{align*}
				t_{1} &= -\tau \ln \left(1 - \frac{V_p}{V_{cc}}\right)
			\end{align*}
			\2 the voltage across the discharging capacitor is 
			\begin{align*}
				v &= V_{p} \exp\left( -\frac{t}{\tau} \right)
			\end{align*}
			\2 the time to discharge to $V_n$ is 
			\begin{align*}
				t_{2} &= -\tau \ln \left(\frac{V_n}{V_{p}}\right)
			\end{align*}
			\2 the voltage across the charging capacitor from $V_n$ to $V_p$ is 
			\begin{align*}
				v &= V_{cc} - (V_{cc} - V_{n})\exp\left( -\frac{t}{\tau} \right) 
			\end{align*}
			\2 the time to charge to from $V_n$ to $V_p$ is
			\begin{align*}
				t_{3} &= -\tau \ln \left(\frac{V_{cc} - V_p}{V_{cc} - V_{n}}\right)
			\end{align*}
			
		\1 For a high pass $RC$ circuit, the output (voltage at capacitor) is simply the input for high frequencies
		\2 at $\omega = 0$, $v_o = 0$ 
		\2 at $\omega \to \infty$, $v_o \to v_{i}$
		\begin{figure}[h]
			\centering
			\begin{circuitikz}[american]
				\draw (0,0) node[circ] {} to[C, l_ = $C$] (2,0) node[circ]{}
				(2,0) to[R, l_ = $R$] (2, -2) node[circ]{} node[ground]{}
				(2,0) -- (4,0) node[ocirc] {}
				(2,-2) -- (4,-2) node[ocirc] {}
				(0,-2) -- (4,-2)
				(4,0) to[open, v = $v_o(t)$] (4,-2) 
				(0,0) to[open, v = $v_i(t)$] (0, -2) node[circ]{}
				;		
			\end{circuitikz}
			\caption{High Pass Circuit}
		\end{figure}
	\end{outline}
	
	\section{Diode and Transistor Switching: Half Wave Rectifier}
	\begin{outline}[enumerate]
		\1 The half wave rectifier has the following circuit
			\begin{figure}[H]
				\centering
				\begin{circuitikz}[american]
					\draw (0,0) node[circ] {} to[D, l_ = $D$] (2,0) node[circ]{}
					(2,0) to[R, l_ = $R$] (2, -2) node[circ]{} node[ground]{}
					(2,0) -- (4,0) node[ocirc] {}
					(2,-2) -- (4,-2) node[ocirc] {}
					(4,0) to[open, v = $v_o(t)$] (4,-2) 
					(0,0) to[sV, v_ = $v_i(t)$] (0, -2) 
					(0,-2) -- (2,-2)
					;		
				\end{circuitikz}
				\caption{Half Wave Rectifier}
			\end{figure}
			\2 the diode is open only when $v_i(t) > v_\text{on}$, hence only the positive voltage are seen by $v_o(t)$
				\3 Average Value $\frac{V_\text{m}}{\pi}$
				\3 $V_\text{rms} = \frac{V_\text{m}}{2}$
				\3 $I_\text{m} = \frac{V_\text{m}}{R_\text{L}}$
				\3 Ripple Factor $\frac{I_\text{rms}}{I_\text{DC}}$
				\3 Efficiency $e = \frac{\text{DC Output Power}}{\text{AC Output Power}}$
			\2 a smoothing capacitor is added (capacitor filter) so that the output waveform does not have a $0$ value: 
			\begin{figure}[H]
				\centering
				\begin{circuitikz}[american]
					\draw (0,0) node[circ] {} to[D, l_ = $D$] (2,0) node[circ]{}
					(2,0) to[R, l_ = $R$] (2, -2) node[circ]{} node[ground]{}
					(2,0) -- (4,0) node[circ]{} to[C, l_ = $C$] (4,-2) node[circ]{}
					(4,0) -- (6,0) node[ocirc] {}
					(4,-2) -- (6,-2) node[ocirc] {}
					(6,0) to[open, v = $v_o(t)$] (6,-2) 
					(0,0) to[sV, v_ = $v_i(t)$] (0, -2) 
					(0,-2) -- (6,-2)
					;		
				\end{circuitikz}
				\caption{Half Wave Rectifier with Capacitor Filter}
			\end{figure}
				\3 $C$ is chosen such that $RC \gg T$ so the exponential seems linear
	\end{outline}

	\section{LC Oscillations}
	\begin{figure}[H]
		\centering
		\begin{circuitikz}[american]
			\draw (0,-2) to[L, l_ = $L$, xscale=-1] (0,0) 
			(0,-2) -- (2,-2)
			(2,-2) to[C, l_=$C$] (2,0)
			(2,0) to[short, i_ = $i$] (0,0)
			(1,0) to[open, v = $v$] (1,-2) node[circ]{} node[ground]{}
			;		
		\end{circuitikz}
		\caption{LC Oscillator}
	\end{figure}	
	\begin{outline}[enumerate]
		\1 most of the time, either $v(0)$ or $i(0)$ are given. 
		\1 by conservation of energy: $\frac{1}{2}Li^2 = \frac{1}{2}Cv^2$
		\1 the differential equations are $v = L\diff{i}{t}$ and $i = -C\diff{v}{t} \implies i = \frac{1}{LC}\diff[2]{i}{t}$
		\1 we have $\omega = \frac{1}{\sqrt{LC}}$
		\1 the solutions have the form 
			\begin{align*}
				i &= I_0\cos\left(\frac{t}{\sqrt{LC}} + \phi \right) \\
				v &= -I_0\sqrt{\frac{L}{C}}\sin\left(\frac{t}{\sqrt{LC}} + \phi \right)
			\end{align*} 
			or 
			\begin{align*}
				v &= V_0\cos\left(\frac{t}{\sqrt{LC}} + \phi \right) \\
				i &= -V_0\sqrt{\frac{C}L{}}\sin\left(\frac{t}{\sqrt{LC}} + \phi \right)
			\end{align*} 
	\end{outline}
	
	
	\section{Source-free RLC Circuits}
	\begin{center}
	\begin{tabular}{|l|c|c|c|c|}
		\hline
		& Characteristic Equation & $(\alpha)$ & $(\omega_0)$ & Damping Factor $(\zeta)$\\ \hline
		Series $i(t)$ & $s^2 + \left(\frac{R}{L}\right)s + \frac{1}{LC} = 0$ & $\alpha = \frac{R}{2L}$ & $\omega_0 = \frac{1}{\sqrt{LC}}$ & $\zeta = \frac{R}{2}\sqrt{\frac{C}{L}}$ \\ \hline
		Parallel $v(t)$ & $s^2 + \left(\frac{1}{RC}\right)s + \frac{1}{LC} = 0$ & $\alpha = \frac{1}{2RC}$ & $\omega_0 = \frac{1}{\sqrt{LC}}$ & $\zeta = \frac{1}{2R}\sqrt{\frac{L}{C}}$ \\ \hline
	\end{tabular}

	\begin{align}
		s_{1,~2} = -\alpha \pm \sqrt{\alpha^2 - \omega_0^2}
	\end{align}

	\begin{tabular}{|l|c|l|l|}
		\hline
		Case & Condition & Characteristic & Roots \\ \hline
		Overdamped & $\zeta > 1$ & Does not oscillate about the steady state value & Real, Distinct \\ \hline
		Underdamped & $\zeta < 1$ & Oscillation with decay envelope & Complex \\ \hline 
		Critacally damped & $\zeta = 1$ & Decays fastest to steady state without oscillation & Real, equal \\ \hline
	\end{tabular}
	\end{center}

	\section{Resonance}
	\begin{outline}[enumerate]
		\1 resonance - the fixed amplitude forcing function produces a response of maximum amplitude
		\1 Series Resonant Conditions:
			\2 $\omega_0 = \frac{1}{\sqrt{LC}}$
			\2 $Z_{in} = R$
			\2 voltage and current of source in phase 
			\2 current magnitude is maximum 
			\2 electric filter 
			\2 the half-power-point frequency is where the power is half: 
				\begin{equation}
					\omega_{1,~2} = \mp \frac{R}{2L} + \sqrt{\left( \frac{R}{2L} \right)^2 + \frac{1}{LC} }
				\end{equation}	
			\2 bandwidth: $BW = \omega_2 - \omega_1 = \frac{R}{L}$
			\2 Quality factor: $Q = \frac{\omega_0}{BW} = \frac{1}{R}\sqrt{\frac{L}{C}}$
			\2 $\omega_0^2 = \omega_1\omega_2$
		\1 Parallel Resonant Conditions:
			\2 $\omega_0 = \frac{1}{\sqrt{LC}}$
			\2 $Y_{in} = \frac{1}{R}$
			\2 voltage and current of source in phase 
			\2 voltage magnitude is maximum 
			\2 electri c filter 
			\2 the half-power-point frequency is where the power is half: 
			\begin{equation}
				\omega_{1,~2} = \mp \frac{1}{2RC} + \sqrt{\left( \frac{1}{2RC} \right)^2 + \frac{1}{LC} }
			\end{equation}	
			\2 bandwidth: $BW = \omega_2 - \omega_1 = \frac{1}{RC}$
			\2 Quality factor: $Q = \frac{\omega_0}{BW} = R\sqrt{\frac{C}{L}}$
			\2 $\omega_0^2 = \omega_1\omega_2$	
		\1 Physical model: $R_1$ series to $L$, $C$ parallel to that series and to $R_2$
			\2 $\omega_0 = \sqrt{\frac{1}{LC} - \left(\frac{R_1}{L} \right)^2 }$	
	\end{outline}

	\newpage

	\section{Laplace Transform Applications}
	Transform of circuit elements: \\
	\begin{outline}[enumerate]
	\1 Resistor:
	\begin{figure}[!htb]
		\centering
		\begin{minipage}{.5\textwidth}
			\centering 
			\begin{circuitikz}[american]
				\draw (0,0) node[ocirc] (pos) {}  to[R, v = $v(t)$, f = $i(t)$] (0,-2) node[ocirc] (neg) {} ;
			\end{circuitikz}
		\end{minipage}%
		\begin{minipage}{0.5\textwidth}
			\centering
			\begin{circuitikz}[american]
				\draw (0,0) node[ocirc] (pos) {}  to[R, v = $V(s)$, f = $I(s)$] (0,-2) node[ocirc] (neg) {} ;
			\end{circuitikz}
		\end{minipage}
	\end{figure}

	\1 Inductor:
	\begin{figure}[!htb]
		\centering
		\begin{minipage}{.25\textwidth}
			\centering 
			\begin{circuitikz}[american]
				\draw (0,0) node[ocirc] (pos) {} to[L, v = $v(t)$, f = $i(t)$] (0,-2) node[ocirc] (neg) {};
			\end{circuitikz}
		\end{minipage}%
		\begin{minipage}{0.3\textwidth}
			\centering
			\begin{circuitikz}[american]
				\draw (0,1.5) to[open, f = $I(s)$] (0,0.5) node[ocirc] (pos) {}  to[L, l = \mbox{$Z(s) = sL$}] (0,-2) to[V, l = $-Li(0^-)$] (0,-4) node[ocirc] (neg) {}
				(-1,1.5) to[open, v = $V(s)$] (-1,-5);
			\end{circuitikz}
		\end{minipage}%
		\begin{minipage}{0.45\textwidth}
			\centering
			\begin{circuitikz}[american]
				\draw (0,-2) to[L, l_ = \mbox{$Y(s) = \frac{1}{sL}$}]  (0,0) 
				(-1,0) to[open,  v^ = $V(s)$] (-1,-2)
				(0,0) -- (3,0) to[I, l = $\frac{i(0^-)}{s}$] (3,-2) -- (0,-2)
				(1.5, 1.5) to[open, f = $I(s)$] (1.5, 0.5) node[ocirc] (pos) {} to[short] (1.5,0)
				(1.5, -2) to[short] (1.5,-2.5)  node[ocirc] (pos) {};
			\end{circuitikz}
		\end{minipage}
	\end{figure}

	\1 Capacitor:
	\begin{figure}[!htb]
		\centering
		\begin{minipage}{.25\textwidth}
			\centering 
			\begin{circuitikz}[american]
				\draw (0,0) node[ocirc] (pos) {} to[C, v = $v(t)$, f = $i(t)$] (0,-2) node[ocirc] (neg) {};
			\end{circuitikz}
		\end{minipage}%
		\begin{minipage}{0.45\textwidth}
			\centering
			\begin{circuitikz}[american]
				\draw (0,-2) to[C, l_ = \mbox{$Y(s) = sC$}]  (0,0) 
				(-1,0) to[open,  v^ = $V(s)$] (-1,-2)
				(0,0) -- (3,0) to[I, l = $Cv(0^-)$, invert] (3,-2) -- (0,-2)
				(1.5, 1.5) to[open, f = $I(s)$] (1.5, 0.5) node[ocirc] (pos) {} to[short] (1.5,0)
				(1.5, -2) to[short] (1.5,-2.5)  node[ocirc] (pos) {};
			\end{circuitikz}
		\end{minipage}%
		\begin{minipage}{0.3\textwidth}
			\centering
			\begin{circuitikz}[american]
				\draw (0,1.5) to[open, f = $I(s)$] (0,0.5) node[ocirc] (pos) {}  to[C, l = \mbox{$Z(s) = \frac{1}{sC}$}] (0,-2) to[V, l = $\frac{v(0^-)}{s}$] (0,-4) node[ocirc] (neg) {}
				(-1,1.5) to[open, v = $V(s)$] (-1,-5);
			\end{circuitikz}
		\end{minipage}		
	\end{figure}
		
	\end{outline}


	\subsection{Method 1}
	\begin{enumerate}
		\item Write the differential equations for the unknown function $x(t)$
		\item Apply Laplace transform on the equation
		\item Use algebraic manipulation for $X(s)$ 
		\item Apply inverse Laplace transform to get $x(t)$
	\end{enumerate}
	
	\subsection{Method 2}
	\begin{enumerate}
		\item Apply Laplace transform on each element
		\item Use DC circuit analysis techniques to write the $s$-domain equations involving $X(s)$
		\item Apply inverse Laplace transform to get $x(t)$
	\end{enumerate}
	
	\textbf{Examples}
	\begin{enumerate}
		\item Resistors: $v = iR \Longleftrightarrow V = IR$, where $V = \mathcal{L}\{v\},~I = \mathcal{L}\{i\}$
		\item Inductors: $v = L\diff{i}{t} \Longleftrightarrow V = L\left[sI - i(0^-)\right] = sLI - LI_0$
		\item Capacitors: $i = C\diff{v}{t} \Longleftrightarrow I = C\left[sV - v(0^-)\right] = sCV - CV_0$ 
	\end{enumerate}
	
	\subsection{$v(t) = V_0u(t)$}
	\begin{enumerate}
		\item Find the equivalent circuit using the Laplace equivalent of the circuit
		\item Solve for $V(s)$. May involve partial fraction decomposition. \href{https://cnx.org/exports/b2e3f8ad-9e60-4421-a343-97e64192ffce\%4015.pdf/partial-fraction-expansion-15.pdf}{This} is a nice guide
		\item Use Inverse Laplace transform to get $v(t)$ 
	\end{enumerate}
	\textbf{Transfer function.} The $s$-domain ratio of the output to input signal. 
	\begin{equation}
	H(s) = \frac{Y(s)}{X(s)}
	\end{equation}
	\begin{center}
	\begin{tabular}{|l|l|l|l|l|l|}
		\hline
		\multicolumn{2}{|l|}{} & $H(s)$ of series RL & \multicolumn{2}{l|}{} & $H(s)$ of parallel RC \\ \hline
		\multirow{3}{*}{Input is $V(t)$} & Output is $i(t)$ & $H(s) = \frac{1}{R + sL}$ & \multirow{3}{*}{Input is $I(t)$} & Output is $v(t)$ & $H(s) = \frac{R}{1 + sRC}$ \\ \cline{2-3} \cline{5-6}
		& Output is $v_{L}(t)$ & $H(s) = \frac{sL}{R + sL}$ & & Output is $i_{C}(t)$ & $H(s) = \frac{sRC}{1 + sRC}$ \\ \cline{2-3}  \cline{5-6}
		& Output is $v_{R}(t)$ & $H(s) = \frac{R}{R + sL}$ & & Output is $i_{R}(t)$ & $H(s) = \frac{1}{1 + sRC}$ \\ \hline 
	\end{tabular}
	\end{center}
		
	\section{Ramp and Sine Response}
	\begin{outline}[enumerate]
		\1 \textbf{Ramp Response:}	
			\2 $\mathcal{L} \left[ r\left( t \right) \right] = \frac{1}{s^{2}}$
		\1 \textbf{Sine Response:}
			\2 use the natural response equation as homogenous solution
			\2 use phasor analysis to find the forced response
			\2 add the two to find the coefficient of the natural response using initial values
			\2 the transient is $0$ for some angle $\atan \left( \frac{\omega L}{R} \right)$ or $\atan \left( \frac{\omega}{RC} \right)$	
	\end{outline}
	\section{Second Order System}
	\begin{outline}[enumerate]
		\1 the following transfer functions are used in RLC circuits:
		\1 Series RLC:	
			\begin{align}
				\frac{V_{L} \left( s \right)}{V_{in} \left( s \right)} &= \frac{ s^{2} }{s^{2} + \left( \frac{R}{L} \right) s + \frac{1}{LC}} \\
				\frac{V_{R} \left( s \right)}{V_{in} \left( s \right)} &= \frac{\left( \frac{R}{L} \right) s }{s^{2} + \left( \frac{R}{L} \right) s + \frac{1}{LC}} \\
				\frac{V_{C} \left( s \right)}{V_{in} \left( s \right)} &= \frac{ \frac{1}{LC} }{s^{2} + \left( \frac{R}{L} \right) s + \frac{1}{LC}}
			\end{align}
		\1 Parallel RLC: 
			\begin{align}
				\frac{I_{L} \left( s \right)}{I_{in} \left( s \right)} &= \frac{ \frac{1}{LC} }{s^{2} + \left( \frac{1}{RC} \right) s + \frac{1}{LC}} \\
				\frac{I_{R} \left( s \right)}{I_{in} \left( s \right)} &= \frac{\left( \frac{1}{RC} \right) s }{s^{2} + \left( \frac{1}{RC} \right) s + \frac{1}{LC}} \\
				\frac{I_{C} \left( s \right)}{I_{in} \left( s \right)} &= \frac{ s^{2} }{s^{2} + \left( \frac{1}{RC} \right) s + \frac{1}{LC}}
			\end{align}
		\1 to determine the damping, rewrite the denominator: $\frac{ \frac{1}{\omega_{0}^{2}} }{ \left( \frac{s}{\omega_{0}} \right)^{2} + 2 \zeta \left( \frac{s}{\omega_{0}} \right) + \frac{1}{\omega_{0}^{2}} }$
	\end{outline}

	\section{Transients in Power Systems}
	\begin{outline}[enumerate]
		Load Energization
		\1 RL Circuit:
			\2 $P = pf * VI = I^{2}R$
			\2 $Q = I^{2}X_{L}$
			\2 $X_{L} = 2\pi f L$ 
		\1 LC (Capacitor Energization)
			\2 $Y_{C} = \frac{Q_{C}}{V^{2}} = 2\pi f C$
			\2 $\omega_{i} = \frac{1}{\sqrt{LC}}$. when $\omega_{i} \gg \omega_{s}$, we assume the source to be DC 
			\2 $V_{s} = V\sqrt{2}$
			\2 $v_{c}\left( t \right) = \left[1 - \cos \left( \omega_{i} t \right) \right]V_{s}$	
			\2 $I_{SC} = \frac{MVA_{SC}}{V}$
			\2 $X_{L} = \frac{V}{I_{SC}} = \frac{V^{2}}{MVA_{SC}} = 2\pi f L$
			\2 $Q_{C} = \frac{V^{2}}{X_{C}} = V^{2} 2\pi f C$
		\1 The Circuit Closing Transient (RL):
			\2 input is $V_{m} \sin \left( \omega t + \theta \right)$
			\2 $i\left( t \right) = \frac{V_{m}}{Z} \left[ \sin \left( \omega t + \theta - \atan \left( \frac{\omega L}{R} \right) \right) - \sin \left( \theta - \atan \left( \frac{\omega L}{R} \right) \right) \exp \left( -\frac{Rt}{L} \right) \right]$
			\2 $Z = \sqrt{R^{2} + \omega^{2}L^{2}}$
			\2 there is no transient when $\theta = \atan \left( \frac{\omega L}{R} \right)$
	\end{outline}
	
	\section{Filter Transient Response and Design}
	\begin{figure}[!htb]
		\centering
		\begin{circuitikz}[american, scale = 0.75]
			\draw (0,0) node[op amp, yscale = -1] (op1) {}
			(op1.+) -- ++(-2,0) coordinate(B)
			(op1.out) to[R, l = $R_{f}$] ++(0,-2) coordinate(C) to [R, l = $R_{1}$] ++(0,-2) node[ground] () {} 
			(op1.out) -- ++(2,0) node[ocirc, label = {right:$v_{o}$}] () {}
			(op1.-) -- ++(-0.5,0) |- (C)
			(B) to[C, l = $C$] ++(0,-3.5) node[ground] () {}
			(B) to[R, l = $R$] ++(-3,0) coordinate(D) to[R, l = $R$] ++(-3,0) node[ocirc, label = {left:$v_{i}$}] () {} 
			(D) |- ++(3,2) to[C, l = $C$] ++(2,0) -| (op1.out)
			; 
		\end{circuitikz}
		\caption{Sallen-Key Lowpass Filter}
	\end{figure}
	\begin{outline}[enumerate]
		\1 $\frac{V_{o} \left( s \right)}{V_{i} \left( s \right)} = \frac{1}{s^{2} \left( RC \right)^{2} + 2s \left( \frac{3 - K}{2} \right) + 1}$
			\2 $K = \frac{R_{f} + R_{1}}{R_{1}}$
			\2 $\omega_{0} = \frac{1}{RC}$
			\2 $\alpha = \frac{3 - K}{2RC}$
			\2 $\zeta = \frac{3 - K}{2} = \frac{\alpha}{\omega_{0}}$
		\1 Analog Filters:
			\2 $R = \frac{k_{1}}{Cf_{c}}$
			\2 $\omega_{0} = \frac{1}{RC} = \frac{f_{c}}{k_{1}}$
			\2 $R_{f} = R_{1}k_{2}$	
			\2 $K = 1 + k_{2}$
			\2 Chebychev filter: has the sharpest roll-off, but allows ripples in the passband
			\2 Butterworth filter: 2nd sharpest roll-off \textit{without} ripples in passband
			\2 Bessel filter: has least ringing and overshoot in step response.
	\end{outline}

	\section{Switching Power Supply}
	
	
\end{document}
