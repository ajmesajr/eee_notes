\documentclass[a4paper,11pt]{article}
\usepackage{ajmesa}
\usepackage[top = 1in, left = 0.5in, right = 0.5in, bottom = 1in]{geometry}

\pagestyle{fancy}
\fancyhf{}
%\rhead{Math 40 - WFY | Julius Basilla}
%\chead{2019-04243}
\lhead{EEE 133 Key concepts and equations}
\cfoot{Page \thepage \hspace{1pt} of \pageref{LastPage}}

\title{\textbf{EEE 133 Key concepts and Equations}}
\author{AJ Mesa Jr.}

\begin{document}
	\maketitle
	\section{Diode Models and Circuits}
	\subsection{History of Electronics}
	\begin{outline}[enumerate]
		\1 1904: John Fleming invented vacuum tubes. First time electron flow is controlled in non-conducting medium
			\2 if the cathode is heated, some electrons can escape 
			\2 if a voltage is applied ($+$) on anode ($-$) on cathode, there will be current 
		\1 1906: the grid was added to control the current between the anode and cathode
			\2 vacuum tubes allowed the development of amplifiers, transmitters, receivers, signal processor
		\1 1946: the ENIAC computer was completed. it could execute 5000 additions per second
		\1 1947: invention of transistor by Shockley, Bardeen, Brattain at the Bell Labs
			\2 they were point-contact transistors
		\1 1948: Schockley invented the BJT. the BJT is monolithic (containted in a single semiconductor crystal)	
		\1 1958: the first Integrated Circuits (IC) was invented by Jack Kilby of Texas Instruments (non-monolithic) 
		\1 1959: the first monolithic IC was developed by Robert Noyce at Fairchild Semiconduction
			\2 uses silicon instead of germanium
			\2 2 interconnected BJT transistor
			\2 $SiO_{2}$ insulator, $Al$ interconnection
			\2 planar IC, 0.06 in diameter
		\1 Bipolar ICs with BJT devices domidated until early 80s	
			\2 S/M/L Scale Integration : 10, 100, 1000 transistors
		\1 1959: MOS ICs started in the 60s after the MOS transistor was develpoed at Bell Labs
			\2 easier to fabricate than bipolar 
			\2 uses less power
			\2 can fit more transistor in the same silicon area
			\2 slower than bipolar IC
			\2 less robust than bipolar IC
		\1 MOS ICs
			\2 early 70s: MOS technology improved in speed and reliability
			\2 first microprocessor: Intel 4004 (1971)
			\2 emergence of VLSI (>10000 trasistors)
			\2 microprocessors evolved into microcontrollers (MCUs) ad system on a chip (SOCs)
		\1 Moore's Law: the transistor density would double every two years
	\end{outline}

	\subsection{Piecewise Linear Diode Models}
	\begin{outline}[enumerate]
		\1 1st Approximation: (when $v_{s} \gg V_{T}$)
			\2 Open when $v_{D} < 0$
			\2 Short when $v_{D} \geq 0$
		\1 2nd Approximation (when above condition not satisfied):
			\2 Open when $v_{D} < V_{T}$
			\2 Voltage of $V_{T}$ when $v_{D} \geq V_{T}$
		\1 3rd Approximation (2nd with resistor $R$ in series)
			\2 Open when $v_{D} < V_{T}$
			\2 Voltage of $V_{T}$ in series with resistance $R$ when $v_{D} \geq V_{T}$	
		\1 To determine diode state: 
			\2 Assume it is conducting 	
			\2 Check direction of current
			\2 If current is from anode to cathode, the assumption is correct and equivalent circuit is valid
			\2 Otherwise, diode should be open
		\1 Another method:
			\2 Replace diode with open circuit 
			\2 Determine the voltage across the diode terminals ($+$) on anode, ($-$) on cathode
			\2 If $v_{D}$ < the needed threshold, assumption is correct 
			\2 Otherwise, diode should be replaced with the appropriate model
	\end{outline}

	\newpage
	\subsection{Practical Diode Circuits}
	\begin{outline}[enumerate]
		\1 Half Wave Rectifier	
		\begin{figure}[!htb]
			\centering
			\begin{circuitikz}[american, full diodes]
				%\ctikzset{bipoles/length=1cm}
				\draw (0,2) to[sV, v_ = $v_{IN}$] (0,0)
				(0,2) to[D, l=$D$] (2,2) to [R, l_=$R$] (2,0) -- (0,0) -- (3,0) node[circ] () {} 
				(2,2) -- (3,2) node[circ] () {} to [open, v=$v_{OUT}$] (3,0)
				;
			\end{circuitikz}
			\caption{Half Wave Rectifier}
		\end{figure}	
		\2 When diode is IDEAL: \\
			$v_{OUT} = \left\{ \begin{array}{ll} 0, & v_{IN} < 0 \\ v_{IN} & v_{IN} \geq 0 \end{array} \right.$
		\2 When diode has constant voltage model: \\
			$v_{OUT} = \left\{ \begin{array}{ll} 0, & v_{IN} < V_{T} \\ v_{IN} - V_{T} & v_{IN} \geq V_{T} \end{array} \right.$
		\1 	Full Wave Rectifier ($v_{1} = v_{2}$)
		\begin{figure}[!htb]
			\centering
			\begin{circuitikz}[american, full diodes]
				%\ctikzset{bipoles/length=1cm}
				\draw (0,2) to[sV, v_ = $v_{1}$] (0,0)
				(0,2) to[D, l=$D_{1}$] (2,2) node[circ] () {} -- (3,2) to [R, l_=$R$] (3,0) -- (0,0) -- (4,0) node[circ] () {} 
				(3,2) -- (4,2) node[circ] () {} to [open, v=$v_{OUT}$] (4,0)
				(0,0) to[sV, v_ = $v_{2}$] (0,-2) to[D, l=$D_{2}$] (2,-2) to[crossing] (2,2)
				;
			\end{circuitikz}
			\caption{Full Wave Rectifier}
		\end{figure}	
			\2 When diode is IDEAL: \\
			$v_{OUT} = \left\vert v_{IN} \right\vert$
			\2 When diode has constant voltage model: \\
			$v_{OUT} = \left\{ \begin{array}{ll} v_{1} - V_{T}, & v_{1} > V_{T} \\ -v_{2} - V_{T}, & v_{2} < -V_{T} \\ 0 & \text{otherwise} \end{array} \right.$
			\2 This requires a transformer with center-tapped secondary ($n_{1} : n_{2} = 1 : 2$)
		
		\newpage
		\1 Full Wave Bridge Rectifier
		\begin{figure}[!htb]
			\centering
			\begin{circuitikz}[american, full diodes]
				\draw (0,3) to[sV, v_=$v_{IN}$] (0,0) 
				(0,3)  -- (3,3) coordinate(top bridge)
				(top bridge) to [D, l=$D_{2}$, *-*] ++(1.5,-1.5) coordinate(right bridge)
				to [D, l=$D_{3}$, *-*, invert] ++(-1.5,-1.5) coordinate (bottom bridge)
				to [D, l=$D_{4}$, *-*, invert] ++(-1.5,1.5) coordinate (left bridge)
				to [D, l=$D_{1}$, *-*] (top bridge)
				(right bridge) -- ++(1,0) coordinate(R+) to[R] ++(0,-2.5) -- ++(1,0) node[circ] () {} coordinate (-)
				(R+) -- ++(1,0) node[circ] () {} to[open, v = $v_{OUT}$] (-) 
				(0,0) to[crossing] (bottom bridge)
				(1.5,1.5) -- (1.5,-1) -- (-)
				;
			\end{circuitikz}
		\caption{Full Wave Bridge Rectifier}
		\end{figure}
			\2 When diode is IDEAL: \\
			$v_{OUT} = \left\vert v_{IN} \right\vert$
			\2 When diode has constant voltage model: \\
			$v_{OUT} = \left\{ \begin{array}{ll} v_{1} - 2V_{T}, & v_{1} > 2V_{T} \\ -v_{2} - 2V_{T}, & v_{2} < -2V_{T} \\ 0 & \text{otherwise} \end{array} \right.$
		
		\1 Positive Clipper	
		\begin{figure}[!htb]
			\centering
			\begin{circuitikz}[american, full diodes]
				%\ctikzset{bipoles/length=1cm}
				\draw (0,3) to[sV, v_ = $v_{IN}$] (0,0)
				(0,3) to [R, l_=$R$] (2,3) to [D, l = $D$] (2,1.5) to [V, v_=$V_{X}$] (2,0) -- (0,0) -- (3,0) node[circ] () {} 
				(2,3) -- (3,3) node[circ] () {} to [open, v=$v_{OUT}$] (3,0)
				;
			\end{circuitikz}
			\caption{Positive Clipper}
		\end{figure}	
			\2 $v_{OUT} = \left\{ \begin{array}{ll} V_{X} + V_{T}, & v_{IN} - V_{X} > V_{T} \\ v_{IN}, & v_{IN} - V_{X} < V_{T}\end{array} \right.$
			
		\newpage	
		\1 Negative Clipper	
		\begin{figure}[!htb]
			\centering
			\begin{circuitikz}[american, full diodes]
				%\ctikzset{bipoles/length=1cm}
				\draw (0,3) to[sV, v_ = $v_{IN}$] (0,0)
				(0,3) to [R, l_=$R$] (2,3) to [D, invert, l = $D$] (2,1.5) to [V, v_=$V_{Y}$,invert] (2,0) -- (0,0) -- (3,0) node[circ] () {} 
				(2,3) -- (3,3) node[circ] () {} to [open, v=$v_{OUT}$] (3,0)
				;
			\end{circuitikz}
			\caption{Negative Clipper}
		\end{figure}	
			\2 $v_{OUT} = \left\{ \begin{array}{ll} -\left( V_{Y} + V_{T} \right), & -V_{Y} - v_{IN} > V_{T} \\ v_{IN}, & -V_{Y} - v_{IN} < V_{T} \end{array} \right.$	
		\1 The positive and negative clippers can be combined 
		\begin{figure}[!htb]
			\centering
			\begin{circuitikz}[american, full diodes]
				%\ctikzset{bipoles/length=1cm}
				\draw (0,3) to[sV, v_ = $v_{IN}$] (0,0)
				(0,3) to [R, l_=$R$] (2,3) to [D, l = $D_{1}$] (2,1.5) to [V, v_=$V_{X}$] (2,0) -- (0,0) -- (5,0) node[circ] () {} 
				(4,3) to [D, invert, l = $D_{2}$] (4,1.5) to [V, v_=$V_{Y}$,invert] (4,0)
				(2,3) -- (5,3) node[circ] () {} to [open, v=$v_{OUT}$] (5,0)
				;
			\end{circuitikz}
			\caption{Positive and Negative Clipper}
		\end{figure}	
			\2 $v_{OUT} = \left\{ \begin{array}{ll} V_{X} + V_{T}, & v_{IN} - V_{X} > V_{T} \\ -\left( V_{Y} + V_{T} \right), & -V_{Y} - v_{IN} > V_{T} \\ v_{IN}, & \text{otherwise} \end{array} \right.$
			
		\1 Peak Detector	
		\begin{figure}[!htb]
			\centering
			\begin{circuitikz}[american, full diodes]
				%\ctikzset{bipoles/length=1cm}
				\draw (0,2) to[sV, v_ = $v_{IN}$] (0,0)
				(0,2) to[D, l=$D$] (2,2) to [cC, l_=$C$] (2,0) -- (0,0) -- (3,0) node[circ] () {} 
				(2,2) -- (3,2) node[circ] () {} to [open, v=$v_{OUT}$] (3,0)
				;
			\end{circuitikz}
			\caption{Peak Detector}
		\end{figure}	
			\2 Let the time $t_{1}$ be the first time the input reaches its maximum value $V_{P}$ and $t_{0}$ be the time to reach $V_{T}$
			\2 When diode is IDEAL: \\
			$v_{OUT} = \left\{ \begin{array}{ll} v_{IN}, & t < t_{1} \\ V_{P}, & t \geq t_1 \end{array} \right.$
			\2 When diode has constant voltage model: \\
			$v_{OUT} = \left\{ \begin{array}{ll} 0, &  0 \leq t < t_{0} \\ v_{IN} - V_{T}, & t_{0} \leq t < t_{1} \\ V_{P} - V_{T} & t_{1} \leq t  \end{array} \right.$
		
		\1 Peak Detector with Load Resistor (Ideal Diode)
		\begin{figure}[!htb]
			\centering
			\begin{circuitikz}[american, full diodes]
				%\ctikzset{bipoles/length=1cm}
				\draw (0,2) to[sV, v_ = $v_{IN}$] (0,0)
				(0,2) to[D, l=$D$] (2,2) to [cC, l_=$C$, v^ = $v_{C}$] (2,0) -- (0,0) -- (5,0) node[circ] () {} 
				(4,2) to[R, l_=$R$] (4,0)
				(2,2) -- (4,2) -- (5,2) node[circ] () {} to [open, v=$v_{OUT}$] (5,0)
				;
			\end{circuitikz}
			\caption{Peak Detector with load resistor}
		\end{figure}	
		\2 Let the time $t_{1}$ be the first time the input reaches its maximum value $V_{P}$ and the period be $T$. When the capacitor voltage decays, let time $t_{2}$ be when $v_{C} = v_{IN}$ again
		\2 $v_{OUT} = \left\{ \begin{array}{ll} v_{IN}, & t < t_{1} \\ V_{P}\exp \left( -\frac{t - t_{1}}{RC} \right), & t_{1} \leq t < t_2 \\ v_{IN}, & t_{2} \leq t < t_{1} + T \\ V_{P}\exp \left( -\frac{t - (t_{1} + T)}{RC} \right), & t_{1} + T \leq t < t_2 + T \\ \vdots & \end{array} \right.$
	
		\1 Negative Clamper	(ideal diode)
		\begin{figure}[!htb]
			\centering
			\begin{circuitikz}[american, full diodes]
				%\ctikzset{bipoles/length=1cm}
				\draw (0,2) to[sV, v_ = $v_{IN}$] (0,0)
				(0,2) to[cC, l=$C$, v = $v_{c}$] (2,2) to [D, l_=$D$] (2,0) -- (0,0) -- (3,0) node[circ] () {} 
				(2,2) -- (3,2) node[circ] () {} to [open, v=$v_{OUT}$] (3,0)
				;
			\end{circuitikz}
			\caption{Negative Clamper}
		\end{figure}	
			\2 Let the time $t_{1}$ be the first time the input reaches its maximum value $V_{P}$
			\2 This is similar to the peak detector, only that $v_{OUT} = v_{IN} - v_{C}$ where $v_{C}$ follows the characteristic of the peak detector
			\2 $v_{C} = \left\{ \begin{array}{ll} v_{IN}, & t < t_{1} \\ V_{P}, & t_{1} \leq t \end{array} \right.$
			\2 $v_{OUT} = \left\{ \begin{array}{ll} 0, & t < t_{1} \\ v_{IN} - V_{P}, & t \geq t_1 \end{array} \right.$
			
		\1 Positive Clamper	(ideal diode)
		\begin{figure}[!htb]
			\centering
			\begin{circuitikz}[american, full diodes]
				%\ctikzset{bipoles/length=1cm}
				\draw (0,2) to[sV, v_ = $v_{IN}$] (0,0)
				(0,2) to[cC, l=$C$, v = $v_{c}$] (2,2) to [D, l_=$D$, invert] (2,0) -- (0,0) -- (3,0) node[circ] () {} 
				(2,2) -- (3,2) node[circ] () {} to [open, v=$v_{OUT}$] (3,0)
				;
			\end{circuitikz}
			\caption{Positive Clamper}
		\end{figure}	
			\2 Let the period be $T$
			\2 $v_{C} = \left\{ \begin{array}{ll} 0, & 0 \leq t < T/2 \\ v_{IN}, & T/2 \leq t < 3T/4 \\ -V_{P}, & 3T/4 \leq t \end{array} \right.$	
			\2 $v_{OUT} = \left\{ \begin{array}{ll} v_{IN}, & 0 \leq t < T/2 \\ 0, & T/2 \leq t < 3T/4\\ v_{IN} + V_{P}, & 3T/4 \leq t \end{array} \right.$
	\end{outline}

	\subsection{Exponential Diode Model}
	\begin{outline}[enumerate]
		\1 Shockley's Diode Equation 
		\begin{equation}
			i_{D} = I_{S}\left[ \exp \left( \frac{v_{D}}{\eta V_{T} } \right) - 1\right]
		\end{equation}
		\2 $v_{D}$: diode voltage with positive at anode
		\2 $i_{D}$: diode current from anode to cathode
		\2 $I_{S}$: reverse saturation current due to minority carriers. $10^{-15} \text{A} < I_{s} < 10^{-9} \text{A}$
		\2 $\eta$: ideality factor, $1 < \eta < 2$; close to 1 for well fabricated diodes
		\2 $V_{T}$: thermal voltage, $V_{T} = \frac{kT_{K}}{e} \approx \frac{T_{K}}{11600}$
		\1 for typical operating voltages (forward biased): $\exp \left( \frac{v_{D}}{\eta V_{T} } \right) \gg 1 \implies i_{D} \approx I_{S} \exp \left( \frac{v_{D}}{\eta V_{T} } \right)$
		\1 for typical reverse bias voltages, $\exp \left( \frac{v_{D}}{\eta V_{T} } \right) \ll 1 \implies i_{D} \approx - I_{S}$	
	\end{outline}

	\subsection{Small Signal Model}
	\begin{outline}[enumerate]
		\1 $v_{D} = V_{D} + v_{d}$ 
		\1 $i_{D} = I_{D} + i_{d}$. $I_{D}$ is calculated using large signal analysis
		\1 if the amplitude of $v_{d}$ is very small, the diode characteristic equation is approximately a line with conductance $g_{d} = \diffp{i_{D}}{{v_{D}}} = \frac{I_{S}}{\eta V_{T}} \exp \left( \frac{v_{D}}{\eta V_{T}} \right) \approx \frac{I_D}{\eta V_{T}} \implies r_{d} = \frac{\eta V_{T}}{I_{D}}$
	\end{outline}

	\subsection{Zener Diode}
	\begin{figure}[!htb]
	\centering
		\begin{circuitikz} [american, full diodes]
			\draw (0,0)node[circ] () {} to[zzD, v = $v_{D}$, invert, f = $i_{D}$] (2,0) node[circ] () {}
			;
		\end{circuitikz}
	\end{figure}
	\begin{outline}[enumerate]
		\1 has 3 modes of operations: \\
		\begin{center}
		\begin{tabular}{|l|l|l|} 
			\hline
			Operation & Condition & Equivalent \\ \hline
			Forward-biased & $v_{D} \leq V_{T}$ & $v_{D} = -V_{T}$ \\ \hline
			Non-conducting & $-V_{T} < v_{D} < V_{Z}$ & open circuit \\ \hline
			Zener region & $v_{D} \geq V_{Z}$ & $v_{D} = V_{Z}$ \\ \hline
		\end{tabular}
		\end{center}
	\end{outline}
	
	\newpage
	\section{PN Juncion and BJT Operation}
	\subsection{Semiconductor Fundamentals}
	\begin{outline}[enumerate]
		\1 The Silicon (14) Atom: 	
			\2 has 4 valence electrons
			\2 the semiconductor lattice has bounded valence electrons at $0$K
			\2 energy bands: electrons can occupy discrete energy levels 
			\2 there is a \textbf{band gap} between the valence band and conduction band 
			\2 at higher temperatures, an electron can gain enough energy to break the bonds 
			\2 when an electron leaves a bond, it leaves a positively charged \textbf{hole}
		\1 Groups by electrical properties: 
			\2 insulators: huge band gap 
			\2 conductor: conduction band and valence overlap
			\2 semiconductor: small band gap that can be easily reached
		\1 Semiconductors: 
			\2 narrow band gap 
			\2 has 2 groups:
				\3 intrinsic: just enough valence electrons to complete the matrix.  either has group IV semiconductors (Si and Ge) or III-V compound (GaAs)		
				\3 extrinsic: produced by doping intrinsic semiconductors
					\4 N-type: dopants have excess electrons. energy level just below the conduction band
					\4 P-type: dopants have less electrons. energy level just above the valence 
		\1 Fermi level: hypothetical energy level where there is $50\%$ probability that the level is occuppied by an electron at any given time 			
			\2 intrinsic semiconductor: midway between $E_{C}$ and $E_{V}$
			\2 extrinsic semiconductor: 
				\3 N-type: closer to $E_{C}$
				\3 P-type: closer to $E_{V}$
	\end{outline}

	\subsection{Carrier Actions in Semiconductors}
	\begin{outline}[enumerate]
		\1 Charge Carriers: the electrons and holes. units in coulombs (C)
		\1 Intrinsic semiconductor: $n = p$ equal concentration of electrons and holes
		\1 N-type semiconductors:
			\2 $n > p$
			\2 majority charge carriers: electrons
			\2 minority charge carriers: holes
		\1 P-type semiconductors:
			\2 $p > n$	
			\2 majority charge carriers: holes
			\2 minority charge carriers: electrons
		\1 Current: rate of movement of charge past a point or region. units in ampere (A). $+$ is the direction of the hole/opposite the electron
		\1 Mobility: ability of charge carrier to move. $\mu_{e} > \mu_{h}$
		\1 Factors that affect mobility:
			\2 Scattering - the random motion of charges. One mechanism is the \textit{lattice scattering} when electrons bump with the vibrating lattice or other electrons and change directions
			\2 another scattering mechanism is \textit{impurity scattering} when dopant ions deflect charge carriers
			\2 Temperature vs Scattering vs Mobility. At low temperatures, impurity scattering dominates (and with low mobility). at high temperatures, lattice scattering dominates (and low mobility). there is a temperature in the middle with maximum mobility. (like a parabolic log-log graph)
			\2 Doping vs Scattering vs Mobility. more doping = more impurity scattering 
			\2 scattering has a net current of $0$ 
		\1 There are 3 primary carrier actions:
			\2 Generation and Recombination: generation = electron gains enough energy to break a covalent bond and leaves a hole. recombination = moving electron moves close to a hole and experience attractive force 
			\2 Diffusion: electrons move from higher concentration/temperature to lower concentration/temperature. $+$ diffusion is parallel to positive charges. becomes $0$ when there is uniform distribution of charge carriers 
			\2 Drift: caused by applied external voltage. modeled by a constant drift velocity $v_{d} = \mu E$ (directly related to the electric field with the mobility as constant)
				\3 $I_n = -Aqnv_\text{n-drift}$
				\3 $I_p = Aqpv_\text{p-drift}$
 	\end{outline}
 
 	\subsubsection{PN Junction}
 	\begin{outline}[enumerate]
 		\1 PN Junction: formed when in a single silicon crystal, one region is doped with donor dopants (N-type) and the other with acceptor dopants (P-type). PN Junction is the boundary.
 		\1 Depletion Region: happens at the boundary when the electrons diffuse and recombine with the holes at the P-type region. This results to a potential difference (and thus an electric field) across the depletion region (points from N to P)
 			\2 drift current of minority carriers are drifted by this electric field
 		\1 There are 3 junction operations:
 			\2 Open-circuit (equilibrium condition):
 				\3 the diffusion current $I_{D}$ is limited by the built in voltage $V_{0}$ and equal to the drift current $I_{S}$
 			\2 Forward Bias ($+$ on the P-type)
 				\3 the $+$ potential pushes the holes in the P-type and the depletion region in the P-side decreases 
 				\3 the $-$ potential pushes the electrons in the N-type and the depletion region in this side also decreases 
 				\3 the diffusion current $I_{D}$ dominates, but the temperature dependent $I_{S}$ remain the same
 				\3 the net current $I = I_{D} - I_{S}$ in the direction from P to N
 				\3 when $V_{F} \geq V_{o}$, the depletion region disappears and $I \approx I_{D}$
 			\2 Reverse Bias ($+$ on the N-type)
 				\3 the $+$ potential attracts the electrons in the N-type, the depletion region becomes larger
 				\3 the $-$ potential attracts the holes in the P-type, the depletion region becomes larger
 				\3 the voltage $V_{o}$ increases so $I_{D}$ decreases
 				\3 net current $I = I_{D} - I_{S}$ from N to P
 				\3 if $I_{D}$ becomes very small, $I \approx -I_{S}$	
 		\1 The energy band diagram will align the Fermi levels, thus bending the $E_{C}$ and $E_{V}$ with the maximum deviation $V_{0}$
 			\2 at forward bias, $V_{0}$ decreases so the bending energy will also decrease and the Fermi level will not be aligned anymore ($E_{FN}$ is higher) 
 			\2 at reverse bias, $V_{0}$ increases so the bending energy will also inrease and the Fermi level will not be aligned anymore ($E_{FN}$ is lower)
 		\1 The PN Junction implements a Diode. anode at P, cathode at N	
 			\2 the Shockley Equation summarizes the 3 conditions for the PN Junction operations
 			\2 Diode resistivity: 
 				\3 at higher temperature, more electrons are freed and resistivity decreases
 				\3 higher doping density lowers the resistivity
 	\end{outline}
	
	\newpage
	\section{BJTs as Amplifiers}
	\subsection{DC Characteristics and Operating Modes}
	\begin{center}
		\begin{tabular}{|l|l|l|}
			\hline
			Mode of Operation & EB Junction & CB Junction \\ \hline
			Cut-off & Reverse-biased & Reverse-biased \\ \hline
			Active & Forward-biased & Reverse-biased \\ \hline
			Saturation & Forward-biased & Forward-biased \\ \hline
			Reverse Active & Forward-biased & Reverse-biased \\ \hline
		\end{tabular}
	\end{center}
	\begin{outline}[enumerate]
		\1 Cutoff Mode Relationships:
			\2 $I_{B} = I_{C} = I_{E} = 0$
		\1 Active Mode Relationships:	
			\2 $I_{B} = \frac{I_{S}}{\beta}  \left[ \exp \left( \frac{eV_{BE}}{kT} \right) - 1 \right] \left[ 1 + \frac{V_{CE}}{V_{A}} \right]$	
			\2 $I_{C} = \beta I_{B} = I_{S} \left[ \exp \left( \frac{eV_{BE}}{kT} \right) - 1 \right] \left[ 1 + \frac{V_{CE}}{V_{A}} \right]$
			\2 $I_{E} = \frac{1}{\alpha} I_{C} = \left( \beta + 1 \right) I_{B}$
			\2 $\alpha = \frac{\beta}{\beta + 1} \Longleftrightarrow \beta = \frac{\alpha}{1 - \alpha}$
			\2 In most cases, $V_{A} \gg V_{CE}$ so the factor is approximately $1$
			\2 Model: 
			\begin{figure}[!htb]
				\centering
				\begin{minipage}{0.5\linewidth}
					\centering
					\begin{circuitikz}[american]
						\draw (-0.5,-0.0) node[ocirc, label={above:$B$}] () {} to[short, f = $I_{B}$] (1,-0.0);
						\draw (1,-0.0) to[V] (1,-2.0);
						\draw (3,-0.0) to[cI, l_ = $\beta I_{B}$] (3,-2.0);
						\draw (4.5,-0.0) node[ocirc, label={above:$C$}] () {} to[short, f_ = $I_{C}$] (3,-0.0);
						\draw (-0.5,-2.0) node[ocirc] () {} to[short] (4.5,-2.0) node[ocirc] () {};
						\draw (2,-2.0) to[short, f = $I_{E}$] (2,-3.5) node[ocirc, label={below:$E$}] () {};
						\draw (-0.5,-0.0) to[open, v = $V_{BE}$] (-0.5,-2.0);
						\draw (4.5,-0.0) to[open, v = $V_{CE}$] (4.5,-2.0);
					\end{circuitikz}
				\caption{NPN Transistor Active Mode}
				\end{minipage}%
				\begin{minipage}{0.5\linewidth}
					\centering
					\begin{circuitikz}[american]
						\draw (-0.5,-0.0) node[ocirc, label={below:$B$}] () {} to[short, f< = $I_{B}$] (1,-0.0);
						\draw (1,2.0) to[V] (1,-0.0);
						\draw (3,2.0) to[cI, l_ = $\beta I_{B}$] (3,-0.0);
						\draw (4.5,0.0) node[ocirc, label={below:$C$}] () {} to[short, f_< = $I_{C}$] (3,0.0);
						\draw (-0.5,2.0) node[ocirc] () {} to[short] (4.5,2.0) node[ocirc] () {};
						\draw (2,2.0) to[short, f< = $I_{E}$] (2,3.5) node[ocirc, label={above:$E$}] () {};
						\draw (-0.5,2.0) to[open, v = $V_{EB}$] (-0.5,-0.0);
						\draw (4.5,2.0) to[open, v = $V_{EC}$] (4.5,-0.0);
					\end{circuitikz}
					\caption{PNP Transistor Active Mode}
				\end{minipage}
			\end{figure}
		\1 Saturation Mode Relationships
			\2 $I_{C,~sat} = \beta_{forced}I_{B}$
			\2 $V_{BE} = 0.7~V$, $V_{EB} = 0.7~V$
			\2 $V_{CE} = 0.3~V$, $V_{EC} = 0.3~V$ edge of saturation
			\2 deep saturation is normally $0.2~V$
			\begin{figure}[!htb]
				\centering
				\begin{minipage}{0.5\linewidth}
					\centering
					\begin{circuitikz}[american]
						\draw (-0.5,-0.0) node[ocirc, label={above:$B$}] () {} to[short, f = $I_{B}$] (1,-0.0);
						\draw (1,-0.0) to[V] (1,-2.0);
						\draw (3,-0.0) to[V] (3,-2.0);
						\draw (4.5,-0.0) node[ocirc, label={above:$C$}] () {} to[short, f_ = $I_{C}$] (3,-0.0);
						\draw (-0.5,-2.0) node[ocirc] () {} to[short] (4.5,-2.0) node[ocirc] () {};
						\draw (2,-2.0) to[short, f = $I_{E}$] (2,-3.5) node[ocirc, label={below:$E$}] () {};
						\draw (-0.5,-0.0) to[open, v = $V_{BE}$] (-0.5,-2.0);
						\draw (4.5,-0.0) to[open, v = $V_{CE}$] (4.5,-2.0);
					\end{circuitikz}
					\caption{NPN Transistor Saturation Mode}
				\end{minipage}%
				\begin{minipage}{0.5\linewidth}
					\centering
					\begin{circuitikz}[american]
						\draw (-0.5,-0.0) node[ocirc, label={below:$B$}] () {} to[short, f< = $I_{B}$] (1,-0.0);
						\draw (1,2.0) to[V] (1,-0.0);
						\draw (3,2.0) to[V] (3,-0.0);
						\draw (4.5,0.0) node[ocirc, label={below:$C$}] () {} to[short, f_< = $I_{C}$] (3,0.0);
						\draw (-0.5,2.0) node[ocirc] () {} to[short] (4.5,2.0) node[ocirc] () {};
						\draw (2,2.0) to[short, f< = $I_{E}$] (2,3.5) node[ocirc, label={above:$E$}] () {};
						\draw (-0.5,2.0) to[open, v = $V_{EB}$] (-0.5,-0.0);
						\draw (4.5,2.0) to[open, v = $V_{EC}$] (4.5,-0.0);
					\end{circuitikz}
					\caption{PNP Transistor Saturation}
				\end{minipage}
			\end{figure}
	\end{outline}

	\subsection{BJT Biasing Circuits}
	\begin{outline}[enumerate]
		\1 Q point: fixed characteristics of the transistor
			\2 active region for amplifier
			\2 cut-off/saturation for switches 
		\1 Biasing: application of DC voltages to establish the operating point 	
		\1 Transistor Parameter variations on Temperature:
			\2 leakage current: $I_{Co}(25^{\circ} C) = 1~nA$. doubles every $6^{\circ}$ rise 
			\2 $V_{BE}(25^{\circ} C) = 0.7~V$. drops about $2.2~mV/^{\circ}C$
			\2 $\beta$ doubles with an increase of $80^{\circ}C$ 
		\begin{figure}[!htb]
			\centering
			\begin{minipage}{0.33\linewidth}
				\centering
				\begin{circuitikz}[american, scale = 0.75]
					\draw (2,-3.0)	node[npn](Q2){};
					\draw (Q2.C) to[short] (2,-2.0);
					\draw (Q2.E) to[short] (2,-4.0);
					\draw (Q2.B) to[short] (1,-3.0);
					\draw (2,-2.0) to[R,l_ = $R_{C}$] (2,-0.0);
					\draw (0,-0.0) to[R,l = $R_{B}$] (0,-3.0);
					\draw (0,-3.0) to[short] (1,-3.0);
					\draw (0,-0.0) to[short] (2,-0.0);
					\draw (2,-4.0) to[short] (2,-6.0);
					\draw (2,-6.0) node[ground]{};
					\draw (2,-0.0) to[short] (2,1.0);
					\draw (2,1.0) node[vcc]{VCC};
				\end{circuitikz}
				\caption{Fixed Bias Circuit}
			\end{minipage}%
			\begin{minipage}{0.33\linewidth}
				\centering
				\begin{circuitikz}[american, scale = 0.75]
					\draw (2,-3.0)	node[npn](Q2){};
					\draw (Q2.C) to[short] (2,-2.0);
					\draw (Q2.E) to[short] (2,-4.0);
					\draw (Q2.B) to[short] (1,-3.0);
					\draw (2,-2.0) to[R,l_ = $R_{C}$] (2,-0.0);
					\draw (0,-0.0) to[R,l = $R_{B}$] (0,-3.0);
					\draw (0,-3.0) to[short] (1,-3.0);
					\draw (0,-0.0) to[short] (2,-0.0);
					\draw (2,-4.0) to[R,l = $R_{E}$] (2,-6.0);
					\draw (2,-6.0) node[ground]{};
					\draw (2,-0.0) to[short] (2,1.0);
					\draw (2,1.0) node[vcc]{VCC};
				\end{circuitikz}
				\caption{Emitter Stabilized}
			\end{minipage}%
			\begin{minipage}{0.33\linewidth}
				\centering
				\begin{circuitikz}[american, scale = 0.75]
					\draw (2,-3.0)	node[npn](Q2){};
					\draw (Q2.C) to[short] (2,-2.0);
					\draw (Q2.E) to[short] (2,-4.0);
					\draw (Q2.B) to[short] (1,-3.0);
					\draw (2,-2.0) to[R,l_ = $R_{C}$] (2,-0.0);
					\draw (0,-0.0) to[R,l_ = $R_{1}$] (0,-3.0);
					\draw (0,-3.0) to[short] (1,-3.0);
					\draw (0,-0.0) to[short] (2,-0.0);
					\draw (2,-4.0) to[R,l = $R_{E}$] (2,-6.0);
					\draw (2,-6.0) node[ground]{};
					\draw (2,-0.0) to[short] (2,1.0);
					\draw (2,1.0) node[vcc]{VCC};
					\draw (0,-3.0) to[R,l_=$R_{2}$] (0,-6.0);
					\draw (0,-6.0) node[ground]{};
				\end{circuitikz}
				\caption{Voltage Divier Bias}
			\end{minipage}
		\end{figure}
	\end{outline}

	\newpage
	\subsection{BJT Amplifiers}
	\begin{outline}[enumerate]
		\1 General steps in circuit analysis:
			\2 In the DC part, treat the capacitors as open
			\2 The DC analysis calculates the Q point values 
			\2 In the AC part, treat the capacitors and DC voltage sources as shorts
		\1 Common Emitter Amplifier:
			\2 Inverting amplifier with very huge gains
		\begin{figure}[!htb]
			\centering
			\begin{circuitikz}[american]
				\draw (2,-3.0) node[npn](Q2){};
				\draw (Q2.C) to[short] (2,-2.0);
				\draw (Q2.E) to[short] (2,-4.0);
				\draw (Q2.B) to[short] (1,-3.0);
				\draw (2,-2.0) to[R,l=$R_{C}$] (2,-0.0);
				\draw (0,-0.0) to[R,l=$R_{2}$] (0,-3.0);
				\draw (0,-3.0) to[short] (1,-3.0);
				\draw (0,-0.0) to[short] (2,-0.0);
				\draw (2,-6.0) node[ground]{};
				\draw (2,-0.0) to[short] (2,1.0);
				\draw (2,1.0) node[vcc]{VCC};
				\draw (2,-4.0) to[R,l=$R_{E}$] (2,-6.0);
				\draw (0,-3.0) to[R,l=$R_{1}$] (0,-6.0);
				\draw (0,-6.0) node[ground]{};
				\draw (2,-2.0) to[C,l=$C \to \infty$] (5.5,-2.0) node[ocirc, label={right:$v_{OUT}$}] () {};
				\draw (3.5,-4.0) to[C,l=$C \to \infty$] (3.5,-6.0);
				\draw (0,-3.0) to[C,l=$C \to \infty$] (-2,-3.0);
				\draw (-2,-3.0) to[V,l=$v_{IN}$] (-2,-6.0);
				\draw (5.5,-2.0) to[R,l=$R_{L}$] (5.5,-6.0);
				\draw (-2,-6.0) node[ground]{};
				\draw (3.5,-6.0) node[ground]{};
				\draw (5.5,-6.0) node[ground]{};
				\draw (2,-4.0) to[short] (3.5,-4.0);
				\draw (5.5,-2.0) node[ocirc, label={right:$v_{OUT}$} ] () {};
			\end{circuitikz}
			\caption{CE Amplifier}
		\end{figure}
		\1 Common Base Amplifier:
			\2 has a sudden rise of gain in the vicinity of the $V_{BE}$ threshold
			\begin{figure}[!htb]
				\centering
				\begin{circuitikz}[american]
					\draw (0,-0.0) node[npn, rotate=90, yscale=-1](Q5){};
					\draw (Q5.C) to[short] (1.0,-0.0);
					\draw (Q5.B) to[short] (0,-1.0);
					\draw (Q5.E) to[short] (-1,-0.0);
					\draw (1,-0.0) to[short] (1,1.0);
					\draw (1,1.0) to[R,l=$R_{C}$] (1,3.0);
					\draw (1,3.0) node[vcc]{VCC};
					\draw (0,-1.0) node[ground]{};
					\draw (-1,-0.0) to[short] (-1.5,-0.0);
					\draw (-1.5,-0.0) node[ocirc, label={above:$v_{IN}$}] () {};
					\draw (1,1.0) to[short] (2,1.0);  
					\draw (2,1.0) node[ocirc, label={above:$v_{OUT}$}] () {};
				\end{circuitikz}
			\end{figure}
		
		\newpage
			
		\1 Common Collector Amplifier:
			\2 has unity gain
			\2 is a voltage follower
			\begin{figure}[!htb]
				\centering
				\begin{circuitikz}[american]
					\draw (2,-0.0) node[npn](Q6){};
					\draw (Q6.C) to[short] (2,1.0);
					\draw (Q6.E) to[short] (2,-1.0);
					\draw (Q6.B) to[short] (1,-0.0);
					\draw (0,-0.0) to[short] (1,-0.0);
					\draw (2,1.0) node[vcc]{VCC};
					\draw (2,-1.0) to[R,l=$R_{E}$] (2,-3.0);
					\draw (2,-3.0) node[ground] {};
					\draw (2,-1.0) to[short] (3,-1.0);
					\draw (3,-1.0) node[ocirc, label={above:$v_{OUT}$}] () {};
					\draw (0,-0.0) node[ocirc, label={above:$v_{IN}$}] () {};
				\end{circuitikz}
			\end{figure}
	\end{outline}

	\newpage
	\section{MOSFETs as Amplifiers}
	\subsection{MOSFET Fundamentals}
	\begin{outline}[enumerate]
		\1 FET: an electric field is applied normal to the surface (gate) of a semiconductor that modulates the conductance of the semiconductor
			\2 NMOS: uses electrons as charge carriers
			\2 PMOS: uses holes as charge carriers
		\1 The source and drain terminals are heavily doped (n for NMOS, p for PMOS)
			\2 Enhancement Mode: default off
			\2 Depletion Mode: default on
		\1 Threshold voltage: minimum $V_{GS}$ for the MOSFET to conduct 	
		\1 Saturation Voltage: minimum $V_{DS}$ that causes the channel of the transistor to pinch-off in the drain side due to widening depletion region in the drain bulk junction
		\1 MOSFET Regions of Operation:
		\begin{center}
			\begin{tabular}{|l|l|l|}
				\hline 
			Region of Operation & Conditions (NMOS) & Current $I_{D}$ \\ \hline 
			Cut-off & $V_{GS} < V_{Th,~n}$ & $I_{D} = 0$ \\ \hline 
			\multirow{2}{*}{Linear} & $V_{GS} \geq V_{Th,~n}$ & \multirow{2}{*}{$I_{D} = k_{n} \left( V_{GS} - V_{Th,~n} - \frac{V_{DS}}{2} \right) V_{DS}$} \\ 
			& $V_{DS} < V_{GS} - V_{Th,~n}$ & \\ \hline
			\multirow{2}{*}{Saturation} & $V_{GS} \geq V_{Th,~n}$ & \multirow{2}{*}{$I_{D} = \frac{1}{2}k_{n} \left( V_{GS} - V_{Th,~n} \right)^{2}$} \\ 
			& $V_{DS} \geq V_{GS} - V_{Th,~n}$ & \\ \hline
			Region of Operation & Conditions (PMOS) & Current $I_{D}$ \\ \hline 
			Cut-off & $V_{GS} < V_{Th,~n}$ & $I_{D} = 0$ \\ \hline 
			\multirow{2}{*}{Linear} & $V_{SG} \geq \absolute{V_{Th,~p}}$ & \multirow{2}{*}{$I_{D} = -k_{p} \left( V_{SG} - V_{Th,~p} - \frac{V_{SD}}{2} \right) V_{SD}$} \\ 
			& $V_{SD} < V_{SG} - V_{Th,~p}$ & \\ \hline
			\multirow{2}{*}{Saturation} & $V_{SG} \geq V_{Th,~p}$ & \multirow{2}{*}{$I_{D} = -\frac{1}{2}k_{p} \left( V_{SG} - V_{Th,~p} \right)^{2}$} \\ 
			& $V_{SD} \geq V_{SG} - V_{Th,~p}$ & \\ \hline
			\end{tabular}
		\end{center}
	\end{outline}

	\subsection{MOSFET DC Characterization}
	\begin{outline}[enumerate]
		\1 Steps in solving MOSFET circuits:	
			\2 Identify the terminals
			\2 Solve for voltage terminals
			\2 Use the current equation to ge $I_{D}$. assume saturation if region of operation is unknown
	\end{outline}

	\subsection{MOSFET Amplifier}
	\begin{outline}[enumerate]
		\1 Common Source Amplifier:	
			\2 inverting amplifier
			\2 can have large gains 
		\begin{figure}[!htb]
			\centering
			\begin{minipage}{0.5\linewidth}
				\centering
				\begin{circuitikz}[american, arrowmos]
					\draw (0,-0.0) node[nmos](Q2){};
					\draw (Q2.D) to[short] (0,1.0);
					\draw (Q2.S) to[short] (0,-1.0);
					\draw (Q2.G) to[short] (-1,-0.0);
					\draw (0,1.0) to[R,l=$R_{B}$] (0,3.0);
					\draw (0,3.0) node[vcc]{$V_{DD}$};
					\draw (0,1.0) to[short] (1,1.0);
					\draw (0,-1.0) node[ground]{};
					\draw (-1,-0.0) to[short] (-1.5,-0.0);
					\draw (-1.5,-0.0) node[ocirc, label={above:$v_{IN}$}] () {};
					\draw (1,1.0) node[ocirc, label={above:$v_{OUT}$}] () {};
				\end{circuitikz}
				\caption{NMOS CS Amplifier}
			\end{minipage}%
			\begin{minipage}{0.5\linewidth}
				\centering
				\begin{circuitikz}[american]
					\draw (0,-0.0) node[pmos](Q3){};
					\draw (Q3.S) to[short] (0,1.0);
					\draw (Q3.D) to[short] (0,-1.0);
					\draw (Q3.G) to[short] (-1,-0.0);
					\draw (-1,-0.0) to[short] (-1.5,-0.0);
					\draw (0,-1.0) to[short] (1,-1.0);
					\draw (0,-1.0) to[R,l=$R_{B}$] (0,-3.0);
					\draw (0,1.0) node[vcc]{$V_{DD}$};
					\draw (0,-3.0) node[ground]{};
					\draw (1,-1.0) node[ocirc, label={above:$v_{OUT}$}] () {};
					\draw (-1.5,-0.0) node[ocirc, label={above:$v_{IN}$}] () {};
				\end{circuitikz}
				\caption{PMOS CS Amplifier}
			\end{minipage}%
		\end{figure}
	
		\1 Common Drain Amplifier:
			\2 has a unity gain 
			\2 non-inverting amplifier
		\begin{figure}[!htb]
			\centering
			\begin{minipage}{0.5\linewidth}
				\centering
				\begin{circuitikz}[american]
					\draw (0,-0.0) node[nmos](Q3){};
					\draw (Q3.S) to[short] (0,-1.0);
					\draw (Q3.D) to[short] (0,1.0);
					\draw (Q3.G) to[short] (-1,-0.0);
					\draw (-1,-0.0) to[short] (-1.5,-0.0);
					\draw (0,-1.0) to[short] (1,-1.0);
					\draw (0,-1.0) to[R,l=$R_{B}$] (0,-3.0);
					\draw (0,1.0) node[vcc]{$V_{DD}$};
					\draw (0,-3.0) node[ground]{};
					\draw (1,-1.0) node[ocirc, label={above:$v_{OUT}$}] () {};
					\draw (-1.5,-0.0) node[ocirc, label={above:$v_{IN}$}] () {};
				\end{circuitikz}
				\caption{NMOS CD Amplifier}
			\end{minipage}%
			\begin{minipage}{0.5\linewidth}
				\centering
				\begin{circuitikz}[american, arrowmos]
					\draw (0,-0.0) node[pmos](Q2){};
					\draw (Q2.D) to[short] (0,-1.0);
					\draw (Q2.S) to[short] (0,1.0);
					\draw (Q2.G) to[short] (-1,-0.0);
					\draw (0,1.0) to[R,l=$R_{B}$] (0,3.0);
					\draw (0,3.0) node[vcc]{$V_{DD}$};
					\draw (0,1.0) to[short] (1,1.0);
					\draw (0,-1.0) node[ground]{};
					\draw (-1,-0.0) to[short] (-1.5,-0.0);
					\draw (-1.5,-0.0) node[ocirc, label={above:$v_{IN}$}] () {};
					\draw (1,1.0) node[ocirc, label={above:$v_{OUT}$}] () {};
				\end{circuitikz}
				\caption{PMOS CD Amplifier}
			\end{minipage}%
		\end{figure}	
	
		\newpage
		\1 Common Gate Amplifier:
			\2 non-inevrting amplifier
			\2 can have gain more than $1$
		\begin{figure}[!htb]
			\centering
			\begin{minipage}{0.5\linewidth}
				\centering
				\begin{circuitikz}[american, arrowmos]
					\draw (0,-0.0) node[nmos, rotate=90, yscale = -1](Q2){};
					\draw (Q2.D) to[short] (1,0.0);
					\draw (Q2.G) to[short] (0,-1.0);
					\draw (Q2.S) to[short] (-1,-0.0);
					\draw (1,0.0) to[short] (1,1.0);
					\draw (1,1.0) to[R,l=$R_{B}$] (1,3.0);
					\draw (1,3.0) node[vcc]{$V_{DD}$};
					\draw (1,1.0) to[short] (2,1.0);
					\draw (0,-1.0) node[ground]{};
					\draw (-1,-0.0) to[short] (-1.5,-0.0);
					\draw (-1.5,-0.0) node[ocirc, label={above:$v_{IN}$}] () {};
					\draw (2,1.0) node[ocirc, label={above:$v_{OUT}$}] () {};
				\end{circuitikz}
				\caption{NMOS CG Amplifier}
			\end{minipage}%
			\begin{minipage}{0.5\linewidth}
				\centering
				\begin{circuitikz}[american]
					\draw (0,-0.0) node[pmos, rotate = -90, yscale = -1, xscale = -1](Q3){};
					\draw (Q3.G) to[short] (0,-2.0);
					\draw (Q3.S) to[short] (-1.5,-0.0);
					\draw (Q3.D) to[short] (1.5,-0.0);
					\draw (0,-2.0) node[ground]{};
					\draw (-1.5,-0.0) node[ocirc, label={above:$v_{IN}$}] () {};
					\draw (1.5,-0.0) to[R, l = $R_{B}$] (1.5,-2.0); 
					\draw (1.5,-2.0) node[ground]{};
					\draw (1.5,-0.0) to[short] (2.5,-0.0);
					\draw (2.5,-0.0) node[ocirc, label={above:$v_{OUT}$}] () {};
				\end{circuitikz}
				\caption{PMOS CG Amplifier}
			\end{minipage}%
		\end{figure}	
	\end{outline}
	
	\newpage
	\section{Small Signal Analysis of Transistor Amplifiers}
	\subsection{Two Port Networks}
	\begin{outline}[enumerate]
		\1 one port serves as input, the other as output
		\1 $Z$ parameters: impedance parameters. taken for open circuit of both ports
		\begin{equation*}
			\begin{bmatrix} V_{1} \\ V_{2} \end{bmatrix} = \begin{bmatrix} Z_{1,~1} & Z_{1,~2} \\ Z_{2,~1} & Z_{2,~2} \end{bmatrix} \begin{bmatrix} I_{1} \\ I_{2} \end{bmatrix}
		\end{equation*}
		\1 $Y$ parameters: admittance parameters. taken for short circuit of both ports
		\begin{equation*}
			\begin{bmatrix} I_{1} \\ I_{2} \end{bmatrix} = \begin{bmatrix} Y_{1,~1} & Y_{1,~2} \\ Y_{2,~1} & Y_{2,~2} \end{bmatrix} \begin{bmatrix} V_{1} \\ V_{2} \end{bmatrix}
		\end{equation*}
		\1 $h$ parameters: hybrid parameters. first column is short circuit of Port 2, second column is open circuit for Port 1
		\begin{equation*}
			\begin{bmatrix} V_{1} \\ I_{2} \end{bmatrix} = \begin{bmatrix} h_{1,~1} & h_{1,~2} \\ h_{2,~1} & h_{2,~2} \end{bmatrix} \begin{bmatrix} I_{1} \\ V_{2} \end{bmatrix}
		\end{equation*}
		\1 $g$ parameters: inverse hybrid parameters. tfirst column is open circuit of Port 2, second column is short circuit for Port 1
		\begin{equation*}
			\begin{bmatrix} I_{1} \\ V_{2} \end{bmatrix} = \begin{bmatrix} g_{1,~1} & g_{1,~2} \\ g_{2,~1} & g_{2,~2} \end{bmatrix} \begin{bmatrix} V_{1} \\ I_{2} \end{bmatrix}
		\end{equation*}
		\1 Unilateral Hybrid $\pi$ Network: only has $3$ parameters
			\2 $R_{i},~R_{o},~A_{v}$ for Thevenin Equivalent
			\2 $R_{i},~R_{o},~G_{m}$ for Norton Equivalent
		\begin{figure}[!htb]
			\centering
			\begin{minipage}{0.5\linewidth}
				\centering
				\begin{circuitikz}[american]
					\draw (0,-0.0) to[short, f = $i_{i}$] (2,-0.0);
					\draw (0,-2.0) to[short] (2,-2.0);
					\draw (2,-0.0) to[R,l=$R_{i}$] (2,-2.0);
					\draw (2,-2.0) to[short] (4,-2.0);
					\draw (4,-0.0) to[cV,l=$A_{v}v_{i}$] (4,-2.0);
					\draw (4,-0.0) to[R,l=$R_{o}$] (6,-0.0);
					\draw (4,-2.0) to[short] (7,-2.0);
					\draw (6,-0.0) to[short, f< = $i_{o}$] (7,-0.0);
					\draw (7,-0.0) node[ocirc] () {};
					\draw (0,-0.0) node[ocirc] () {};
					\draw (7,-2.0) node[ocirc] () {};
					\draw (0,-2.0) node[ocirc] () {};
					\draw (0,-0.0) to[open, v = $v_{i}$] (0,-2.0);
					\draw (7,-0.0) to[open, v = $v_{o}$] (7,-2.0);
				\end{circuitikz}
				\caption{Thevenin Equivalent}
			\end{minipage}%
			\begin{minipage}{0.5\linewidth}
				\centering
				\begin{circuitikz}[american]
					\draw (0,-0.0) to[short, f = $i_{i}$] (2,-0.0);
					\draw (0,-2.0) to[short] (2,-2.0);
					\draw (2,-0.0) to[R,l=$R_{i}$] (2,-2.0);
					\draw (2,-2.0) to[short] (4,-2.0);
					\draw (4,-2.0) to[short] (7,-2.0);
					\draw (4,-0.0) to[cI,l=$G_{m}v_{i}$] (4,-2.0);
					\draw (6,-0.0) to[R,l=$R_{o}$] (6,-2.0);
					\draw (4,-0.0) to[short] (6,-0.0);
					\draw (6,-0.0) to[short, f< = $i_{o}$] (7,-0.0);
					\draw (7,-0.0) node[ocirc] () {};
					\draw (0,-0.0) node[ocirc] () {};
					\draw (7,-2.0) node[ocirc] () {};
					\draw (0,-2.0) node[ocirc] () {};
					\draw (0,-0.0) to[open, v = $v_{i}$] (0,-2.0);
					\draw (7,-0.0) to[open, v = $v_{o}$] (7,-2.0);
				\end{circuitikz}
				\caption{Norton Equivalent}
			\end{minipage}%
			\caption{Unilateral Hybrid $\pi$ Network}
		\end{figure}	
	\end{outline}

	\subsection{Small Signal Model}
	\begin{center}
	\begin{tabular}{|l|l|l|}
		\hline 
		Parameter & BJT & MOSFET \\ \hline
		$g_{m}$ Transconductance & $g_{m} = \frac{I_{C}}{V_{T}}$ & $g_{m} = \frac{2I_{D}}{V_{GS} - V_{Th,~n}} = \sqrt{2K_{n}I_{D} \left(1 + \lambda V_{DS}\right)} \approx \sqrt{2K_{n}I_{D}}$ \\ \hline
		$r_{\pi}$ Input Resistance & $r_{\pi} = \frac{\beta}{g_{m}} = \frac{\beta V_{T}}{I_{C}}$ & $r_{\pi} = \infty$ \\ \hline
		$r_{o}$ Output Resistance & $r_{o} = \frac{V_{A} + V_{CE}}{I_{C}} \approx \frac{V_{A}}{I_{C}}$ & $\frac{1}{I_{D}} \left( \frac{1}{\lambda} + V_{DS}\right) \approx \frac{1}{\lambda I_{D}}$ \\ \hline
	\end{tabular}
	\end{center}
	\begin{outline}[enumerate]
		\1 the approximations above are from:
			\2 $V_{A} \gg V_{CE}$
			\2 $\lambda \ll \frac{1}{V_{DS}}$
	\end{outline}
	\begin{figure}[!htb]
		\centering
		\begin{minipage}{0.5\linewidth}
			\centering
			\begin{circuitikz}[american]
				\draw (0,-0.0) to[short] (1,-0.0);
				\draw (1,-0.0) to[open,v=$v_{gs}$] (1,-2.0);
				\draw (1,-2.0) to[short] (5,-2.0);
				\draw (3,-0.0) to[cI,l=$g_{m}v_{gs}$] (3,-2.0);
				\draw (5,-0.0) to[R,l=$r_{o}$] (5,-2.0);
				\draw (3,-0.0) to[short] (5,-0.0);
				\draw (5,-0.0) to[short] (6,-0.0);
				\draw (3,-1.0) to[short] (3,-3.0);
				\draw (1,-0.0) node[circ] () {};
				\draw (1,-2.0) node[circ] () {};
				\draw (0,-0.0) node[ocirc, label={above:$G$}] () {};
				\draw (3,-3.0) node[ocirc, label={below:$S$}] () {};
				\draw (6,-0.0) node[ocirc, label={above:$D$}] () {};
			\end{circuitikz}
			\caption{MOSFET Small Signal Model}
		\end{minipage}%
		\begin{minipage}{0.5\linewidth}
			\centering
			\begin{circuitikz}[american]
				\draw (0,-0.0) to[short] (1,-0.0);
				\draw (1,-0.0) to[R, l = $r_{\pi}$] (1,-2.0);
				\draw (0.5,-0.0) to[open, v = $v_{\pi}$] (0.5,-2.0);
				\draw (1,-2.0) to[short] (5,-2.0);
				\draw (3,-0.0) to[cI,l=$g_{m}v_{\pi}$] (3,-2.0);
				\draw (5,-0.0) to[R,l=$r_{o}$] (5,-2.0);
				\draw (3,-0.0) to[short] (5,-0.0);
				\draw (5,-0.0) to[short] (6,-0.0);
				\draw (3,-1.0) to[short] (3,-3.0);
				\draw (0,-0.0) node[ocirc, label={above:$B$}] () {};
				\draw (3,-3.0) node[ocirc, label={below:$E$}] () {};
				\draw (6,-0.0) node[ocirc, label={above:$C$}] () {};
			\end{circuitikz}
			\caption{BJT Small Signal Model}
		\end{minipage}%
	\end{figure}

	\subsection{Small Signal Analysis}
	\begin{outline}[enumerate]
		\1 Amplifier Parameters:
			\begin{center}
			\begin{tabular}{|l|l|}
				\hline 
				Voltage Gain & $A_{v} = \frac{v_{o}}{v_{i}}$ \\ \hline 
				Open-loop Gain & $A_{vo} = \left. \frac{v_{o}}{v_{i}} \right|_{R_{L} \to \infty}$ \\ \hline
				Input Resistance & $R_{i} = \frac{v_{i}}{i_{i}}$ \\ \hline
				Output Resistance of Amplifier & $R_{o} = \left. \frac{v_{o}}{i_{o}} \right|_{v_{i} \to 0}$ \\ \hline
				Output Resistance of the circuit & $R_{out} = \left. \frac{v_{o}}{i_{o}} \right|_{v_{sig} \to 0}$ \\ \hline
				Overall Gain & $A = \frac{R_{i}}{R_{i} + R_{sig}}A_{v}$ \\ \hline
			\end{tabular}
			\end{center}
		\1 Design considerations:
			\2 $R_{i} \gg R_{sig}$
			\2 $R_{o} \ll R_{L}$
		\1 Solving Amplifier Circuits
			\2 Find Bias and Signal Circuits
			\2 Get the bias (Q point) values	
				\3 capacitors are open
				\3 large signal models
			\2 Get the signal paramaters	
				\3 capacitors are shorts
				\3 independent sources are suppressed
				\3 replace with small signal model	
			\2 Terminal Resistance:
				\3 Consider that a voltage $v_{x}$ and current $i_{x}$	is supplied through the terminal. the terminal resistance is $\frac{v_{x}}{i_{x}}$
	\end{outline}
	
	\section{Amplifier Frequency Response}
\end{document}
