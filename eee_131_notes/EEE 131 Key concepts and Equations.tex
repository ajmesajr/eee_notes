\documentclass[a4paper,11pt]{article}
\usepackage{ajmesa}
\usepackage[top = 1in, left = 0.5in, right = 0.5in, bottom = 1in]{geometry}

\pagestyle{fancy}
\fancyhf{}
%\rhead{Math 40 - WFY | Julius Basilla}
%\chead{2019-04243}
\lhead{EEE 133 Key concepts and equations}
\cfoot{Page \thepage \hspace{1pt} of \pageref{LastPage}}

\title{\textbf{EEE 133 Key concepts and Equations}}
\author{AJ Mesa Jr.}

\begin{document}
	\maketitle
	\section{Diode Models and Circuits}
	\subsection{History of Electronics}
	\begin{outline}[enumerate]
		\1 1904: John Fleming invented vacuum tubes. First time electron flow is controlled in non-conducting medium
			\2 if the cathode is heated, some electrons can escape 
			\2 if a voltage is applied ($+$) on anode ($-$) on cathode, there will be current 
		\1 1906: the grid was added to control the current between the anode and cathode
			\2 vacuum tubes allowed the development of amplifiers, transmitters, receivers, signal processor
		\1 1946: the ENIAC computer was completed. it could execute 5000 additions per second
		\1 1947: invention of transistor by Shockley, Bardeen, Brattain at the Bell Labs
			\2 they were point-contact transistors
		\1 1948: Schockley invented the BJT. the BJT is monolithic (containted in a single semiconductor crystal)	
		\1 1958: the first Integrated Circuits (IC) was invented by Jack Kilby of Texas Instruments (non-monolithic) 
		\1 1959: the first monolithic IC was developed by Robert Noyce at Fairchild Semiconduction
			\2 uses silicon instead of germanium
			\2 2 interconnected BJT transistor
			\2 $SiO_{2}$ insulator, $Al$ interconnection
			\2 planar IC, 0.06 in diameter
		\1 Bipolar ICs with BJT devices domidated until early 80s	
			\2 S/M/L Scale Integration : 10, 100, 1000 transistors
		\1 1959: MOS ICs started in the 60s after the MOS transistor was develpoed at Bell Labs
			\2 easier to fabricate than bipolar 
			\2 uses less power
			\2 can fit more transistor in the same silicon area
			\2 slower than bipolar IC
			\2 less robust than bipolar IC
		\1 MOS ICs
			\2 early 70s: MOS technology improved in speed and reliability
			\2 first microprocessor: Intel 4004 (1971)
			\2 emergence of VLSI (>10000 trasistors)
			\2 microprocessors evolved into microcontrollers (MCUs) ad system on a chip (SOCs)
		\1 Moore's Law: the transistor density would double every two years
	\end{outline}

	\subsection{Piecewise Linear Diode Models}
	\begin{outline}[enumerate]
		\1 1st Approximation: (when $v_{s} \gg V_{T}$)
			\2 Open when $v_{D} < 0$
			\2 Short when $v_{D} \geq 0$
		\1 2nd Approximation (when above condition not satisfied):
			\2 Open when $v_{D} < V_{T}$
			\2 Voltage of $V_{T}$ when $v_{D} \geq V_{T}$
		\1 3rd Approximation (2nd with resistor $R$ in series)
			\2 Open when $v_{D} < V_{T}$
			\2 Voltage of $V_{T}$ in series with resistance $R$ when $v_{D} \geq V_{T}$	
		\1 To determine diode state: 
			\2 Assume it is conducting 	
			\2 Check direction of current
			\2 If current is from anode to cathode, the assumption is correct and equivalent circuit is valid
			\2 Otherwise, diode should be open
		\1 Another method:
			\2 Replace diode with open circuit 
			\2 Determine the voltage across the diode terminals ($+$) on anode, ($-$) on cathode
			\2 If $v_{D}$ < the needed threshold, assumption is correct 
			\2 Otherwise, diode should be replaced with the appropriate model
	\end{outline}

	\newpage
	\subsection{Practical Diode Circuits}
	\begin{outline}[enumerate]
		\1 Half Wave Rectifier	
		\begin{figure}[!htb]
			\centering
			\begin{circuitikz}[american, full diodes]
				%\ctikzset{bipoles/length=1cm}
				\draw (0,2) to[sV, v_ = $v_{IN}$] (0,0)
				(0,2) to[D, l=$D$] (2,2) to [R, l_=$R$] (2,0) -- (0,0) -- (3,0) node[circ] () {} 
				(2,2) -- (3,2) node[circ] () {} to [open, v=$v_{OUT}$] (3,0)
				;
			\end{circuitikz}
			\caption{Half Wave Rectifier}
		\end{figure}	
		\2 When diode is IDEAL: \\
			$v_{OUT} = \left\{ \begin{array}{ll} 0, & v_{IN} < 0 \\ v_{IN} & v_{IN} \geq 0 \end{array} \right.$
		\2 When diode has constant voltage model: \\
			$v_{OUT} = \left\{ \begin{array}{ll} 0, & v_{IN} < V_{T} \\ v_{IN} - V_{T} & v_{IN} \geq V_{T} \end{array} \right.$
		\1 	Full Wave Rectifier ($v_{1} = v_{2}$)
		\begin{figure}[!htb]
			\centering
			\begin{circuitikz}[american, full diodes]
				%\ctikzset{bipoles/length=1cm}
				\draw (0,2) to[sV, v_ = $v_{1}$] (0,0)
				(0,2) to[D, l=$D_{1}$] (2,2) node[circ] () {} -- (3,2) to [R, l_=$R$] (3,0) -- (0,0) -- (4,0) node[circ] () {} 
				(3,2) -- (4,2) node[circ] () {} to [open, v=$v_{OUT}$] (4,0)
				(0,0) to[sV, v_ = $v_{2}$] (0,-2) to[D, l=$D_{2}$] (2,-2) to[crossing] (2,2)
				;
			\end{circuitikz}
			\caption{Full Wave Rectifier}
		\end{figure}	
			\2 When diode is IDEAL: \\
			$v_{OUT} = \left\vert v_{IN} \right\vert$
			\2 When diode has constant voltage model: \\
			$v_{OUT} = \left\{ \begin{array}{ll} v_{1} - V_{T}, & v_{1} > V_{T} \\ -v_{2} - V_{T}, & v_{2} < -V_{T} \\ 0 & \text{otherwise} \end{array} \right.$
			\2 This requires a transformer with center-tapped secondary ($n_{1} : n_{2} = 1 : 2$)
		
		\newpage
		\1 Full Wave Bridge Rectifier
		\begin{figure}[!htb]
			\centering
			\begin{circuitikz}[american, full diodes]
				\draw (0,3) to[sV, v_=$v_{IN}$] (0,0) 
				(0,3)  -- (3,3) coordinate(top bridge)
				(top bridge) to [D, l=$D_{2}$, *-*] ++(1.5,-1.5) coordinate(right bridge)
				to [D, l=$D_{3}$, *-*, invert] ++(-1.5,-1.5) coordinate (bottom bridge)
				to [D, l=$D_{4}$, *-*, invert] ++(-1.5,1.5) coordinate (left bridge)
				to [D, l=$D_{1}$, *-*] (top bridge)
				(right bridge) -- ++(1,0) coordinate(R+) to[R] ++(0,-2.5) -- ++(1,0) node[circ] () {} coordinate (-)
				(R+) -- ++(1,0) node[circ] () {} to[open, v = $v_{OUT}$] (-) 
				(0,0) to[crossing] (bottom bridge)
				(1.5,1.5) -- (1.5,-1) -- (-)
				;
			\end{circuitikz}
		\caption{Full Wave Bridge Rectifier}
		\end{figure}
			\2 When diode is IDEAL: \\
			$v_{OUT} = \left\vert v_{IN} \right\vert$
			\2 When diode has constant voltage model: \\
			$v_{OUT} = \left\{ \begin{array}{ll} v_{1} - 2V_{T}, & v_{1} > 2V_{T} \\ -v_{2} - 2V_{T}, & v_{2} < -2V_{T} \\ 0 & \text{otherwise} \end{array} \right.$
		
		\1 Positive Clipper	
		\begin{figure}[!htb]
			\centering
			\begin{circuitikz}[american, full diodes]
				%\ctikzset{bipoles/length=1cm}
				\draw (0,3) to[sV, v_ = $v_{IN}$] (0,0)
				(0,3) to [R, l_=$R$] (2,3) to [D, l = $D$] (2,1.5) to [V, v_=$V_{X}$] (2,0) -- (0,0) -- (3,0) node[circ] () {} 
				(2,3) -- (3,3) node[circ] () {} to [open, v=$v_{OUT}$] (3,0)
				;
			\end{circuitikz}
			\caption{Positive Clipper}
		\end{figure}	
			\2 $v_{OUT} = \left\{ \begin{array}{ll} V_{X} + V_{T}, & v_{IN} - V_{X} > V_{T} \\ v_{IN}, & v_{IN} - V_{X} < V_{T}\end{array} \right.$
			
		\newpage	
		\1 Negative Clipper	
		\begin{figure}[!htb]
			\centering
			\begin{circuitikz}[american, full diodes]
				%\ctikzset{bipoles/length=1cm}
				\draw (0,3) to[sV, v_ = $v_{IN}$] (0,0)
				(0,3) to [R, l_=$R$] (2,3) to [D, invert, l = $D$] (2,1.5) to [V, v_=$V_{Y}$,invert] (2,0) -- (0,0) -- (3,0) node[circ] () {} 
				(2,3) -- (3,3) node[circ] () {} to [open, v=$v_{OUT}$] (3,0)
				;
			\end{circuitikz}
			\caption{Negative Clipper}
		\end{figure}	
			\2 $v_{OUT} = \left\{ \begin{array}{ll} -\left( V_{Y} + V_{T} \right), & -V_{Y} - v_{IN} > V_{T} \\ v_{IN}, & -V_{Y} - v_{IN} < V_{T} \end{array} \right.$	
		\1 The positive and negative clippers can be combined 
		\begin{figure}[!htb]
			\centering
			\begin{circuitikz}[american, full diodes]
				%\ctikzset{bipoles/length=1cm}
				\draw (0,3) to[sV, v_ = $v_{IN}$] (0,0)
				(0,3) to [R, l_=$R$] (2,3) to [D, l = $D_{1}$] (2,1.5) to [V, v_=$V_{X}$] (2,0) -- (0,0) -- (5,0) node[circ] () {} 
				(4,3) to [D, invert, l = $D_{2}$] (4,1.5) to [V, v_=$V_{Y}$,invert] (4,0)
				(2,3) -- (5,3) node[circ] () {} to [open, v=$v_{OUT}$] (5,0)
				;
			\end{circuitikz}
			\caption{Positive and Negative Clipper}
		\end{figure}	
			\2 $v_{OUT} = \left\{ \begin{array}{ll} V_{X} + V_{T}, & v_{IN} - V_{X} > V_{T} \\ -\left( V_{Y} + V_{T} \right), & -V_{Y} - v_{IN} > V_{T} \\ v_{IN}, & \text{otherwise} \end{array} \right.$
			
		\1 Peak Detector	
		\begin{figure}[!htb]
			\centering
			\begin{circuitikz}[american, full diodes]
				%\ctikzset{bipoles/length=1cm}
				\draw (0,2) to[sV, v_ = $v_{IN}$] (0,0)
				(0,2) to[D, l=$D$] (2,2) to [cC, l_=$C$] (2,0) -- (0,0) -- (3,0) node[circ] () {} 
				(2,2) -- (3,2) node[circ] () {} to [open, v=$v_{OUT}$] (3,0)
				;
			\end{circuitikz}
			\caption{Peak Detector}
		\end{figure}	
			\2 Let the time $t_{1}$ be the first time the input reaches its maximum value $V_{P}$ and $t_{0}$ be the time to reach $V_{T}$
			\2 When diode is IDEAL: \\
			$v_{OUT} = \left\{ \begin{array}{ll} v_{IN}, & t < t_{1} \\ V_{P}, & t \geq t_1 \end{array} \right.$
			\2 When diode has constant voltage model: \\
			$v_{OUT} = \left\{ \begin{array}{ll} 0, &  0 \leq t < t_{0} \\ v_{IN} - V_{T}, & t_{0} \leq t < t_{1} \\ V_{P} - V_{T} & t_{1} \leq t  \end{array} \right.$
		
		\1 Peak Detector with Load Resistor (Ideal Diode)
		\begin{figure}[!htb]
			\centering
			\begin{circuitikz}[american, full diodes]
				%\ctikzset{bipoles/length=1cm}
				\draw (0,2) to[sV, v_ = $v_{IN}$] (0,0)
				(0,2) to[D, l=$D$] (2,2) to [cC, l_=$C$, v^ = $v_{C}$] (2,0) -- (0,0) -- (5,0) node[circ] () {} 
				(4,2) to[R, l_=$R$] (4,0)
				(2,2) -- (4,2) -- (5,2) node[circ] () {} to [open, v=$v_{OUT}$] (5,0)
				;
			\end{circuitikz}
			\caption{Peak Detector with load resistor}
		\end{figure}	
		\2 Let the time $t_{1}$ be the first time the input reaches its maximum value $V_{P}$ and the period be $T$. When the capacitor voltage decays, let time $t_{2}$ be when $v_{C} = v_{IN}$ again
		\2 $v_{OUT} = \left\{ \begin{array}{ll} v_{IN}, & t < t_{1} \\ V_{P}\exp \left( -\frac{t}{RC} \right), & t_{1} \leq t < t_2 \\ v_{IN}, & t_{2} \leq t < t_{1} + T \\ V_{P}\exp \left( -\frac{t}{RC} \right), & t_{1} + T \leq t < t_2 + T \\ \vdots & \end{array} \right.$
	
		\1 Negative Clamper	(ideal diode)
		\begin{figure}[!htb]
			\centering
			\begin{circuitikz}[american, full diodes]
				%\ctikzset{bipoles/length=1cm}
				\draw (0,2) to[sV, v_ = $v_{IN}$] (0,0)
				(0,2) to[cC, l=$C$, v = $v_{c}$] (2,2) to [D, l_=$D$] (2,0) -- (0,0) -- (3,0) node[circ] () {} 
				(2,2) -- (3,2) node[circ] () {} to [open, v=$v_{OUT}$] (3,0)
				;
			\end{circuitikz}
			\caption{Negative Clamper}
		\end{figure}	
			\2 Let the time $t_{1}$ be the first time the input reaches its maximum value $V_{P}$
			\2 This is similar to the peak detector, only that $v_{OUT} = v_{IN} - v_{C}$ where $v_{C}$ follows the characteristic of the peak detector
			\2 $v_{C} = \left\{ \begin{array}{ll} v_{IN}, & t < t_{1} \\ V_{P}, & t_{1} \leq t \end{array} \right.$
			\2 $v_{OUT} = \left\{ \begin{array}{ll} 0, & t < t_{1} \\ v_{IN} - V_{P}, & t \geq t_1 \end{array} \right.$
			
		\1 Positive Clamper	(ideal diode)
		\begin{figure}[!htb]
			\centering
			\begin{circuitikz}[american, full diodes]
				%\ctikzset{bipoles/length=1cm}
				\draw (0,2) to[sV, v_ = $v_{IN}$] (0,0)
				(0,2) to[cC, l=$C$, v = $v_{c}$] (2,2) to [D, l_=$D$, invert] (2,0) -- (0,0) -- (3,0) node[circ] () {} 
				(2,2) -- (3,2) node[circ] () {} to [open, v=$v_{OUT}$] (3,0)
				;
			\end{circuitikz}
			\caption{Positive Clamper}
		\end{figure}	
			\2 Let the period be $T$
			\2 $v_{C} = \left\{ \begin{array}{ll} 0, & 0 \leq t < T/2 \\ v_{IN}, & T/2 \leq t < 3T/4 \\ -V_{P}, & 3T/4 \leq t \end{array} \right.$	
			\2 $v_{OUT} = \left\{ \begin{array}{ll} v_{IN}, & 0 \leq t < T/2 \\ 0, & T/2 \leq t < 3T/4\\ v_{IN} + V_{P}, & 3T/4 \leq t \end{array} \right.$
	\end{outline}

	\subsection{Exponential Diode Model}
	\begin{outline}[enumerate]
		\1 Shockley's Diode Equation 
		\begin{equation}
			i_{D} = I_{S}\left[ \exp \left( \frac{v_{D}}{\eta V_{T} } \right) - 1\right]
		\end{equation}
		\2 $v_{D}$: diode voltage with positive at anode
		\2 $i_{D}$: diode current from anode to cathode
		\2 $I_{S}$: reverse saturation current due to minority carriers. $10^{-15} \text{A} < I_{s} < 10^{-9} \text{A}$
		\2 $\eta$: ideality factor, $1 < \eta < 2$; close to 1 for well fabricated diodes
		\2 $V_{T}$: thermal voltage, $V_{T} = \frac{kT_{K}}{e} \approx \frac{T_{K}}{11600}$
		\1 for typical operating voltages (forward biased): $\exp \left( \frac{v_{D}}{\eta V_{T} } \right) \gg 1 \implies i_{D} \approx I_{S} \exp \left( \frac{v_{D}}{\eta V_{T} } \right)$
		\1 for typical reverse bias voltages, $\exp \left( \frac{v_{D}}{\eta V_{T} } \right) \ll 1 \implies i_{D} \approx - I_{S}$	
	\end{outline}

	\subsection{Small Signal Model}
	\begin{outline}[enumerate]
		\1 $v_{D} = V_{D} + v_{d}$ 
		\1 $i_{D} = I_{D} + i_{d}$. $I_{D}$ is calculated using large signal analysis
		\1 if the amplitude of $v_{d}$ is very small, the diode characteristic equation is approximately a line with conductance $g_{d} = \diffp{i_{D}}{{v_{D}}} = \frac{I_{S}}{\eta V_{T}} \exp \left( \frac{v_{D}}{\eta V_{T}} \right) \approx \frac{I_D}{\eta V_{T}} \implies r_{d} = \frac{\eta V_{T}}{I_{D}}$
	\end{outline}

	\subsection{Zener Diode}
	\begin{figure}[!htb]
	\centering
		\begin{circuitikz} [american, full diodes]
			\draw (0,0)node[circ] () {} to[zzD, v = $v_{D}$, invert, f = $i_{D}$] (2,0) node[circ] () {}
			;
		\end{circuitikz}
	\end{figure}
	\begin{outline}[enumerate]
		\1 has 3 modes of operations: \\
		\begin{center}
		\begin{tabular}{lll} 
			Operation & Condition & Equivalent \\
			Forward-biased & $v_{D} \leq V_{T}$ & $v_{D} = -V_{T}$ \\
			Non-conducting & $-V_{T} < v_{D} < V_{Z}$ & open circuit \\
			Zener region & $v_{D} \geq V_{Z}$ & $v_{D} = V_{Z}$
		\end{tabular}
		\end{center}
	\end{outline}
	
	
	\section{PN Juncion and BJT Operation}
	\section{BJTs as Amplifiers}
	\section{MOSFETs as Amplifiers}
	\section{Small Signal Analysis of Transistor Amplifiers}
	\section{Amplifier Frequency Response}
\end{document}